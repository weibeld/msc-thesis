At the beginning of the 1960s, a Swiss logician named Julius Richard Büchi at Michigan University was looking for a way to prove the decidability of the satisfiability of monadic second order logic with one successor (S1S). Büchi applied a trick that truly founded a new paradigm in the application of logic to theoretical computer science. He thought of interpretations of a S1S formula as infinitly long words of a formal language and designed a type of finite state automaton that accepts such a word if and only if the interpretation it represents satisfies the formula. After proving that every S1S formula can be translated to such an automaton and vice versa (Büchi's Theorem), the satisfiabilty problem of an S1S formula could be reduced to testing the non-emptiness of the corresponding automaton.

This special type of finite state automaton was later called Büchi automaton.

\subsection{Context of Study}

Büchi automata are finite state automata that process words of infinite length, so called \om-words. A word is accepted if there is at least one accepting state that is visited infinitely often. The complement of a Büchi automaton $A$ is denoted by $\cl1{A}$ and accepts a word if and only if it is not accepted by $A$.

For a word $\alpha \in \Sigma^\omega$:

\begin{quote}
\centering
Word $\alpha$ accepted by $A$ $\Longleftrightarrow$ word $\alpha$ not accepted by $\cl1{A}$
\end{quote}

So, there is no word of the set of possible words $\Sigma^\omega$ that is accepted or not accepted by \textit{both} of $A$ and its complement $\cl1{A}$.

One of the main applications of Büchi complementation is language containment of \om-regular languages. This means to find out whether all the words of a language $L_1$ are also contained in another language $L_2$, formally written as $L_1 \subseteq L_2$.

The way this problem is solved is by testing $L_1 \cap \cl1{L_2} = \varnothing$. That is, one takes the intersection of the first (contained) language and the \textit{complement} of the second (containing) language, and tests whether this intersection is empty. If yes, then all the words of $L_1$ are also contained in $L_2$, and $L_1 \subseteq L_2$ is true. If no, then there is at least one word of $L_1$ that is not contained in $L_2$, and $L_1 \subseteq L_2$ is false.

If we represent languages as automata, then the problem becomes $L(A_1) \subseteq L(A_2)$, which is solved by testing $L(A_1) \cap L(\cl1{A_2}) = \varnothing$. That means, we have to complement the automton $A_2$.

Knowing that the complementation of Büchi automata is used for testing language containment of \om-regular languages, raises the question what is langauge containment of \om-regular langauges used for? In the following, we will briefly present one important application of language containment, namely the language containment approach to automata-theoretic model checking.

Automata-theoretic model checking is an approach to model checking, which in turn is an approach to formal verification. Formal verification means the use of mathemtical techniques for proving the correctness of a system (software of hardware) with respect to a specification~\cite{2007_vardi_model_checking}. A typical example is to prove that a system has no deadlocks.

\begin{figure}[htb]
\centering
\ModelChecking
\caption{Branch of the family of formal verification appraoches.}
\label{model_checking}
\end{figure}

The langauge containment approach to automata-theoretic model checking works as follows. The system, whose correctness is to prove, is represented as a Büchi automaton, say $S$. This automaton $S$ defines a language $L(S)$, and the words of this language correspond to all the possible computation traces that the system can produce.

On the other hand, the property that the system must satisfy (for example, deadlock-freeness) is represented as another Büchi automaton, say $P$. The words of the language $L(P)$ correspond to all the possible computation traces that satisfy the property.

With these two representations in place, the verification step is done by testing whether $L(S) \subseteq L(P)$. That is, whether the language defined by the system automaton $S$ is contained in the language defined by the property automaton $P$. As we have seen, this problem is solved by testing whether $L(S) \cap L(\cl1{P}) = \varnothing$. This is in turn algorithmically done by testing $L(S \cap \cl1{P}) = \varnothing$, which includes the following three steps:

\begin{enumerate}
\item Construct the complement $\cl1{P}$ of the property automaton $P$
\item Construct the intersection of $S$ and $\cl1{P}$, that is, $A_{S,\cl1{P}} = S \cap \cl1{P}$
\item Test whether $A_{S,\cl1{P}}$ is empty
\end{enumerate}

If the emptiness test is positive, then $L(S) \subseteq L(P)$ is true, and the system satsifies the property (for example, deadlock-freeness) with all its possible computation traces. If the emptiness test is negative, then $L(S) \subseteq L(P)$ is false, and there is at least one computation trace of the system that violates the property.

As can be seen from these three steps, the verification problem is reduced to three operations on non-deterministic Büchi automata: (1) complementation, (2) intersection, and (3) emptiness testing. It turns out that intersection and emptiness testing have efficient solutions~\cite{1996_vardi}, whereas 


Where Büchi complementation is used and why it is important

\begin{itemize}
\item What are Büchi automata (very short)
\item What is Büchi complementation (very short)
\item Application of Büchi complementation (longer)
  \begin{itemize}
  \item Main usage in anguage containment: $L_1 \subseteq L_2$ done by testing whether $L_1 \cap \cl1{L_2} = \varnothing$
    \begin{itemize}
    \item In terms of automata: $L(A) \subseteq L(A^\prime)$ by testing $L(A) \cap L(\cl1{A^\prime}) = \varnothing$, that is $A^\prime$ must be complemented
    \end{itemize}
  \item Important application of language containment: language containment approach to automata-theoretic model checking
    \begin{itemize}
    \item Model system as Büchi automaton $M$
    \item Represent specification properties as Büchi automaton $P$
    \item Test $L(M) \subseteq L(P)$, that is, $L(M) \cap L(\cl1{P}) = \varnothing$
    \item Need to complement Büchi automaton $P$, which is very difficult. Alternatives:
      \begin{itemize}
      \item Specify property as deterministic Büchi automaton (complementation is easy). Disadvantage: DBW less expressive, less intuitive, larger automata
      \item Directly represent negation of properites as Büchi automaton. Disadvantage: difficult
      \item Different approach to automata-theoretic model checking: specify properties as LTL formulas, negate them, and translate to Büchi automaton, model system as labelled transition system and translate to Büchi automaton (used by SPIN). Disadvantage: LTL is less expressive than Büchi automata
      \end{itemize}
    \item Importance of more efficient Büchi complementation: so far no tool includes complementation of Büchi automata~\cite{2007_vardi_model_checking}
    \end{itemize}
  \end{itemize}
\end{itemize}


A Büchi complementation construction takes as input a Büchi automaton $A$ and produces as output another Büchi automaton $B$ which accepts the complement language of the input automaton $A$. Complement language denotes the ``contrary'' language, that is, $B$ must \emph{accept} (over a given alphabet) every word that $A$ \emph{does not} accept, and must in turn \emph{not accept} every word that $A$ \emph{accepts}.

Büchi automata are finite automata (that is, having a finite number of states) which operate on infinite words (that is, words that ``never end''). Operating on infinite words, they belong thus to the category \om-automata. An important application of Büchi automata is in model checking which is a formal system verification technique. There, they are used to represent both, the description of the system to be checked for the presence of a correctness property, and (the negation of) this correctness property itself.

In one approach to model checking, the correctness property is directly specified as a Büchi automaton
One approach to model checking requires that the Büchi automaton representing the correctness property is complemented. It is here that the problem of Büchi complementation has one of its practical applications. 

\subsection{Stating the problem, reason the research is worth tackling}

Regarding the state complexity of Büchi complementation constructions, only the worst-case state growths have been investigated. However, they are a poor guide to actual peformance of constructions~\cite{2011_tsai}. Need for empirical complexity investigations to see the \textit{actual} performance of complementation constructions.

The complementation of non-deterministic Büchi automata is hard. It has been proven to have an exponential lower bound in the number of generated states [cite]. That is, the number of states of the output automaton is, in the worst case, an exponential function of the number of states of the input automaton. However, since the introduction of Büchi automata in the 1960's, significant process in reducing the complexity (in other words, the degree of exponentiality) of the Büchi complementation problem has been made. Some numbers [list complexities of the different constructions].

\subsection{Aim and Scope}

Aim: empirical performance investigation of a specific Büchi complementaiton construction, comparison with other constructions

Scope: two test sets, relatively small automata, no real world or ``typical'' examples,

% Significance: advance knowledge, contribute to solutin of practical problem, novel use of a procedure or technique?

\subsection{Overview}


% 2. The Büchi complementation problem
% 2.1 Preliminaries
% Used naming convention (NBW, DBW, etc.)
%   2.1.1 Büchi automata
%     - Definition
%     - DBW vs. NBW
%     - Equivalence with omega-regular languages (NBW)
%     - DBW weaker than NBW (proof)
%   2.1.2 Other omega-automata
%     - Muller, Rabin, Streett, Parity
%     - McNaughton's Theorem (NBW = DMW)
%     - Complete picture of equivalences
%   2.1.3 Complementation of Büchi automata
%     - NBW closed under complementation (DBW not, proof)
%     - Example NFA/DFA
%     - Complementation of DBW (Kurshan)
%   2.1.4 Complexity of Büchi Complementation
%     - Notion of state blow-up
%     - Lower bounds for the state blow-up
% 2.2 Review of Büchi Complementation Constructions
% 2.3 Empirical Performance Investigations