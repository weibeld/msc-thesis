At the beginning of the 1960's, a Swiss logician named Julius Richard Büchi was looking for a way to decide the satisfiability of formulas of the monadic second order logic with one successor (S1S). In his quest, Büchi observed that an S1S formula can be represented by a certain type of finite state automaton that runs on infinite words, such that this automaton accepts a word if and only if the corresponding interpretation satisfies the formula. The proof of this equivalence between S1S formulas and this type of autoamtaon, which is known as \textit{Büchi's Theorem}, led Büchi to his desired decision procedure: to test whether an S1S formula $\varphi$ is satisfiable, translate it to an equivalent automaton $A$, and test whether $A$ is empty (that is, accepts no words at all). If $A$ is empty, then $\varphi$ is unsatisfiable, if $A$ is non-empty, then $\varphi$ is satsifiable.~\cite{buchi1960decision}

The type of automaton that Büchi used for solving this logical problem is called \textit{Büchi automaton}. The application of Büchi automata to logic, that was established by Büchi, had a large impact on other fields, especially model checking, which is a technique of formal verification. In particular, Büchi automata allow to solve the model checking question automata-theoretically, which has many advantages~\cite{1996_vardi}.

However, there is one operation on Büchi automata that is giving a ``headache'' to the research community since the introduction of Büchi automata more than 50 years ago, namely the problem of \textit{complementation}. Algorithms for carrying out this operation, although possible\footnote{Büchi himself has proved that Büchi automata are closed under complementation~\cite{buchi1960decision}.}, turn out to be very complex, in many cases too complex for practical application. Yet, Büchi complementation has a practical application in the automata-theoretic approach to model checking. This discrepancy led to an ongoing quest for finding more efficient \textit{Büchi complementation constructions}, and generally better understanding the complexity of Büchi complementation. The work in this thesis is situated in this area of research.

In this introductory chapter, we will first 


% - automata-theoretic approach to logic
% - Büchi's Theorem
%   - Proof includes proving that Büchi automata are closed under complementation (most difficult part).


% At the beginning of the 1960s, a Swiss logician named Julius Richard Büchi at Michigan University was looking for a way to prove the decidability of the satisfiability of monadic second order logic with one successor (S1S). Büchi applied a trick that truly founded a new paradigm in the application of logic to theoretical computer science. He thought of interpretations of a S1S formula as infinitly long words of a formal language and designed a type of finite state automaton that accepts such a word if and only if the interpretation it represents satisfies the formula. After proving that every S1S formula can be translated to such an automaton and vice versa (Büchi's Theorem), the satisfiabilty problem of an S1S formula could be reduced to testing the non-emptiness of the corresponding automaton.

% This special type of finite state automaton was later called Büchi automaton.

\section{Context}
\label{1_context}
%% Büchi automata and Büchi complementation

\subsection{Büchi Automata and Büchi Complementation}

Büchi automata are finite state automata that process words of infinite length, so called \om-words. If $\Sigma$ is the alphabet of a Büchi automaton, then the set of all the possible \om-words that can be generated from this alphabet is denoted by $\Sigma^\omega$. A word $\alpha \in \Sigma^\omega$ is accepted by a Büchi automaton if it results in at least one run that contain infinitely many occurrences of at least one accepting state. A run of a Büchi automaton on a word is an infinite sequence of states. Deterministic Büchi automata have at most one run for each word in $\Sigma^\omega$, whereas non-deterministic Büchi automata may have multiple runs for a word.

The complement of a Büchi automaton $A$ is another Büchi automaton\footnote{Büchi automata are closed under complementation. This has been proved by Büchi~\cite{buchi1960decision}, who, to this end, described the first Büchi complementation construction in history.} denoted by $\cl1{A}$. Both, $A$ and $\cl1{A}$, share the same alphabet $\Sigma$. Regarding a word $\alpha \in \Sigma^\omega$, the relation between an automaton and its complement is as follows:

\begin{quote}
\centering
$\alpha$ accepted by $A$ $\Longleftrightarrow$ $\alpha$ not accepted by $\cl1{A}$
\end{quote}

That is, all the words of $\Sigma^\omega$ that are \textit{accepted} by an automaton are \textit{rejected} by its complement, and all the words of $\Sigma^\omega$ that are \textit{rejected} by an automaton are \textit{accepted} by its complement. In other words, there is no single word of $\Sigma^\omega$ that is  accepted or rejected by \textit{both} of an automaton and its complement.

A Büchi complementation construction is an algorithm that, given a Büchi automaton, creates the complement of this Büchi automaton. The difficulty of this operation depends on whether the input automaton is determinstic or non-deterministic. The complementation of deterministic Büchi automata is ``easy'' and can be done in polynomial time and linear space~\cite{Kurshan198759}. The complementation of non-deterministic Büchi automata, however, is very complex. The understanding and reduction of its complexity is a domain of active research and lies at the centre of this thesis.

Consequently, when in the following we talk about ``Büchi complementation'', we specifically mean the complementation of \textit{non-determinstic} Büchi automata. The main problem with the complexity of Büchi complementation is the so-called state growth or state complexity (sometimes also called state blow-up or state explosion). This is the number of states of the output automaton in relation to the number of states of the input automaton. In simple words, Büchi complementation constructions produce complements that may be very, very large.

This inhibits the practial applicaion of Büchi complementation, because in this case the limited computing and time resources may not be high enough to accommodate for this high complexity. In the following subsections we highlight an important application that Büchi complementation has in practice, and thereby motivate the research on Büchi complementation and of this thesis.

% The complementation of Büchi automata, in particular non-deterministic Büchi automata, is commonly known as ``Büchi complementation'' or the ``Büchi complementation problem''. It is a very complex problem because it exhibits a very high state growth, which is sometimes even called state explosion (in the following, we will use the terms state growth, state explosion, and state complexity interchangeably). State growth denotes the relation of the number of states of a complement $\cl1{A}$ (output of the complementation construction) to the number of states of the automaton $A$ (input to the complementation construction). This relation is for worst-case automata exponential, even for an ideal complementation construction\footnote{Yan proved in 2007 a lower bound for the worst-case state growth of Büchi complementation of $(0.76n)^n$, where $n$ is the number of states of the initial automaton~\cite{DBLP:journals/corr/abs-0802-1226}.}. Even though the state growth that existing complementation constructions produce for many non-worst-case automata is not nearly as high as the worst case, it may still be very high. This is a serious problem, because Büchi complementation has important practical applications (as we will see next), and it is the reason that the quest for more efficient and more practical Büchi complementation constructions is still an active research topic today.


\subsection{Applications of Büchi Complementation}

\subsubsection{Language Containment of \om-Regular Languages}

Büchi complementation is used for testing language containment of \om-regular languages. The \om-regular languages are the class of formal languages that is equivalent to non-deterministic Büchi automata\footnote{Note that deterministic Büchi automata have a lower expressivity than non-deterministic Büchi automata, and are equivalent to only a subset of the \om-regular languages.}. At this point, we briefly describe the language containment in general, before in turn describing an application of the language containment problem in the next subsection.

Given two \om-regular languages $L_1$ and $L_2$ over alphabet $\Sigma^\omega$ the language containment problem consists in testing whether $L_1 \subseteq L_2$. This is true if all the words of $L_1$ are also in $L_2$, and false if $L_1$ contains at least one word that is not in $L_2$. The way this problem is algorithmically solved is by testing $L_1 \cap \cl1{L_2} = \varnothing$. Here, $\cl1{L_2}$ denotes the complement language of $L_2$, which means $\cl1{L_2}$ contains all the words of $\Sigma^\omega$ that are \textit{not} in $L_2$. The steps for testing $L_1 \cap \cl1{L_2} = \varnothing$ are the following:

\begin{itemize}
\item Create the complement language $\cl1{L_2}$ of $L_2$
\item Create the intersection language $L_{1,\cl1{2}}$ of $L_1$ and $\cl1{L_2}$
\item Test whether $L_{1,\cl1{2}}$ is empty (that is, contains no words at all)
\end{itemize}

Thus, the language containment problem is reduced to three operations on languages, \textit{complementation}, \textit{intersection}, and \textit{emptiness testing}. The common way to work with formal languages is not to handle the languages themselves, but more compact structures that represent them, such as automata. In the case of \om-regular languages, these are non-deterministic Büchi automata.

For solving $L_1 \subseteq L_2$, one thus works with two Büchi automata $A_1$ and $A_2$ that represent $L_1$ and $L_2$, respectively. The problem then becomes $L(A_1) \subseteq L(A_2)$, and equivalently, $L(A_1) \cap \cl1{L(A_2)} = \varnothing$. This is automata-theoretically solved as $\text{\textsf{empty}}(A_1 \cap \cl1{A_2})$, which includes the three following steps:

\begin{itemize}
\item Construct the complement automaton $\cl1{A_2}$ of $A_2$
\item Construct the intersection automaton $A_{1,\cl1{2}}$ of $A_1$ and $A_2$
\item Test whether $A_{1,\cl1{2}}$ is empty (that is, accepts no words at all)
\end{itemize}

If the final emptiness test on automaton $A_{1,\cl1{2}}$ is true, then $L_1 \subseteq L_2$ is true, and if the emptiness test is false, then $L_1 \subseteq L_2$ is false. In this way, the language containment problem of \om-regular languages is reduced to three operations of \textit{complementation}, \textit{intersection}, and \textit{emptiness testing} of non-deterministic Büchi automata. Thus, Büchi complementation is an intergral part of language contaiment of \om-regular languages.

%  This means, we have to create the intersection language, say $L_{1,\cl1{2}}$ of $L_1$ and the complement of $L_2$, and then test whether $L_{1,\cl1{2}}$ is empty (that is, contains no words at all). If $L_{1,\cl1{2}}$ is empty, then there is no word of $L_1$ that is not also in $L_2$, and $L_1 \subseteq L_2$ is true. If $L_{1,\cl1{2}}$ is non-empty, then there is at least one word of $L_1$ that is not in $L_2$, and $L_1 \subseteq L_2$ is false.

% With this procedure, we in fact reduce the language containment problem to three operations on languages: complementation, intersection, and emptiness testing. By translating the languages $L_1$ and $L_2$ to equivalent automata $A_1$ and $A_2$, and mainpulating the automata instead of the languages, the problem becomes $L(A_1 \cap \cl1{A_2}) = \varnothing$. That is, we complement $A_2$, create the intersection automaton $A_{1,\cl1{2}}$ of $A_1$ and the complement of $A_2$, and test whether the language of $A_{1,\cl1{2}}$ is empty, which is done by directly testing the automaton $A_{1,\cl1{2}}$ for emptiness.

% In this way, we reduce the language containment problem of \om-regular languages to three operations on non-deterministic Büchi automata: complementation, intersection, and emptiness testing. Büchi complementation is thus an integral part of the language contaiment problem. However, this does not yet answer our initial question of what is a \textit{concrete} and \textit{practical} application of Büchi complementation. To answer this question, we will in the following describe one important application of langauge containment of \om-regular languages.


% One of the main applications of Büchi complementation is language containment of \om-regular languages. This means to find out whether all the words of a language $L_1$ are also contained in another language $L_2$, formally written as $L_1 \subseteq L_2$.

% The way this problem is solved is by testing $L_1 \cap \cl1{L_2} = \varnothing$. That is, one takes the intersection of the first (contained) language and the \textit{complement} of the second (containing) language, and tests whether this intersection is empty. If yes, then all the words of $L_1$ are also contained in $L_2$, and $L_1 \subseteq L_2$ is true. If no, then there is at least one word of $L_1$ that is not contained in $L_2$, and $L_1 \subseteq L_2$ is false.

% If we represent languages as automata, then the problem becomes $L(A_1) \subseteq L(A_2)$, which is solved by testing $L(A_1) \cap L(\cl1{A_2}) = \varnothing$. That means, we have to complement the automton $A_2$.


\subsubsection{Automata-Theoretic Model Checking via Language Containment}
In the last subsection, we have seen that Büchi complementation is used for testing language containment of \om-regular languages. In this subsection, we will see what in turn language containment of \om-regular languages is used for. To this end, we describe one important application of it, namely the language containment approach to automata-theoretic model checking. In the following, we first describe basic working of this technique in general, and then point out the significance that Büchi complementation bears for it.

\subsubsection{Basics}
The language containment approach to automata-theoretic model checking is an approach to automata-theoretic model checking, which is an approach to model checking, which in turn is an approch to formal verification~\cite{2007_vardi_model_checking}. Figure~\ref{model_checking} shows the branch of the family of formal verification techniques that contains the language containment approach to automata-theoretic model checking.

\begin{figure}[htb]
\centering
\ModelChecking
\caption{Branch of the family of formal verification techniques that contains the language containment approach to automata-theoretic model checking.}
\label{model_checking}
\end{figure}

Formal verification is the use of mathematical techniques for proving the correctness of a system (software of hardware) with respect to a specified property~\cite{2007_vardi_model_checking}. A typical example is to verify that a program is deadlock-free (in which case the property would be ``deadlock-freeness''). In general, formal verification techniques consist of the following three parts~\cite{huth2004logic}:

\begin{enumerate}
\item A framework for modelling the system to verify
\item A framework for specifying the property to be verified
\item A verification method for testing whether the system satisfies the property
\end{enumerate}

For the language containment approach to automata-theoretic model checking, the frameworks for points 1 and 2 are Büchi automata representing \om-regular languages. The verification method (point 3) is testing language containment of the system automaton's language in the property automaton's language. In some more detail, this works as follows.~\cite{1996_vardi}\cite{2007_vardi_model_checking}

The system $s$ to verify is modelled as a Büchi automaton, say $S$. This Büchi automaton represents an \om-regular language $L(S)$, and each word of $L(S)$ represents in turn a possible \textit{computation trace} of the system. A computation trace is an infinite\footnote{The infinity of computation traces suggests that this type of formal verification (and model checking in general) is used for systems that are not expected to terminate and  may run indefinitely. This type of systems is called \textit{reactive} systems. They contrast with systems that are expected to terminate and produce a result. For this latter type of systems other formal verification techniques than model checking are used. See for example~\cite{huth2004logic} and~\cite{ben2012mathematical} for works that cover the formal verification of both types of systems.} sequence of ``situations'' that the system is in at any point in time. Such a situation consists of a finite amount of information of, for example, the values of variables, registers, or buffers. The observation that such a trace can be represented as a word of an \om-regular langauges comes from the fact that it can be represented intuitively as a linear Kripke structure (which in turn is an interpretation for a temporal logic formula that can also be used to represent computations), which in turn can be represented by a word of a language whose alphabet is ranges over the powerset of the atomic propositions of the Kripke strucutre. A work that explains these intimate relations between computation, temporal logic, formal languages, and automata in more detail is~\cite{1996_vardi}. In simple words, the language $L(S)$, represented by the system automaton $S$, represents everything that the system \textit{can} do.

Similarly, a property $p$ to be verified is represented as a Büchi automaton, say $P$, which represents the \om-regular language $L(P)$, whose words represent computation traces. These computation traces are all the computations of a system like $s$ that satisfy the property $p$. If for example $p$ is ``deadlock-freeness'', then the words of $L(P)$ represent all the possible computation traces that do \textit{not} contain a deadlock. In this way, the language $L(P)$ represents everything that the system is \textit{allowed} to do, with respect to a certain property.

The verification step is finally done by testing $L(S) \subseteq L(P)$. If this is true, then everything that the system \textit{can} do is \textit{allowed} to do, and the system satisfies the property $p$. If the language containment test is false, then the system \textit{can} do a computation that is \textit{not allowed} to do, and the system does not satisify the property $p$.

Summarising, the language containment approach to automata-theoretic model checking requires language containment of \om-regular languages, which, as we have seen, requires Büchi complementation. In the following, we will highlight the particular importance of Büchi complementation for this type of formal verification.

%   with each state being a situation of the system.  . The observation that computation traces can be represented as words of an \om-regular language is in turn based on temporal logic. 

%  Each word of the language $L(S)$ of $S$ corresponds to a possible computation trace of the system $s$. A computation trace is an infinite sequence of ``combinations of properties'' of the system. Such properties can be for example variable values or statuses of individual processes. Each element of a computation trace corresponds to a point in time during the execution of the system. The language $L(S)$ represents thus everything that the system \textit{can} do.

% A property $p$ to be verified is represented as a non-deterministic Büchi automaton, say $P$. The words of the language $L(P)$ of $P$ also represent computation traces. In particular, the language $L(P)$ represents all the possible computation traces that do satisfy the property $p$. If for example $p$ is ``deadlock-freeness'', then $L(P)$ contains all the possible computation traces of a system like $s$ that are deadlock-free. In other words, the language $L(P)$ represents everything that a system is \textit{allowed} to do, with respect to a property $p$.

% The verification step then consists in testing $L(S) \subseteq L(P)$, that is, whether everything that the system \textit{can} do is contained in everything that the system is \textit{allowed} to do. If this is true, then every possible computation trace of the system satisfies the property $p$, because it is contained in $L(P)$. If $p$ means deadlock-freeness, then we can conclude that the system is deadlock-free. If the language containment test returns a negative result, then there must be at least one computation trace of the system that does not satisfiy the property $p$, because it is not in $L(P)$. In that case, this computation trace (however improbable it is), for example, leads to a deadlock, and we have to conclude that the system is not deadlock-free.

% This short description points out the application of language containment in formal verification, and, as we know, language containment of \om-regular languages requires Büchi complementation. In the following, we will briefly mention some interesing points about the specific role of Büchi complementation in automata-theoretic model checking.

\subsection{Significance of Büchi Complementation}
The complexity of Büchi complementation makes the just described model checking approach nearly inapplicable in practice~\cite{1995_tasiran}. According to~\cite{2007_vardi_model_checking}, there are so far no tools that include this approach. This is unfortunate, because the other Büchi operations for language containment, intersection and emptiness testing, have highly efficient solutions~\cite{2007_vardi_model_checking} (cf.~\cite{1996_vardi}), and thus Büchi complementation is the only bottleneck. Existing practical applications are thus forced to circumvent the need for Büchi complementation. This is possible, has however certain disadvantages as we will see in the following.

One way to circumvent the complementation of non-deterministic Büchi automata is to specify the property as a deterministic Büchi automaton~\cite{1995_tasiran}\cite{2007_vardi_model_checking}. As we have mentioned, the complementation of deterministic Büchi automata has an efficient solution. The disadvantage of this approach is, however, that the property automaton may become exponentially larger, and that it is generally more complicated and less intuitive to represent a language as a deterministic automaton~\cite{1995_tasiran}.

Another way is to use a different model checking approach altogether, which leads us back to the essence of model checking. In basic model checking, the property to be verified is represented as a formula $\varphi$ of a temporal logic (typically LTL). The system to verify is represnted as a Kripke structure $K$, which serves as an interpretation of the formula $\varphi$. The verification step consists in checking whether $K$ is a \textit{model} of $\varphi$. An interpretation $K$ is a model of a formula $\varphi$, if every state of the interpretation \textit{satisifes} the formula, written as $K \models \varphi$. This test of modelhood is the reason that this verification approach is called \textit{model checking}~\cite{1996_vardi}.

The modelhood test can be done automata-theoretically without the need for Büchi complementation~\cite{2007_vardi_model_checking} (in Figure~\ref{model_checking}, this would be a sibling to the language containment approach). The Kripke structure $K$ is translated to a Büchi automaton $A_K$. The formula $\varphi$ is negated and translated to the Büchi automaton $A_{\neg\varphi}$. Finally, one tests $\text{\textsf{empty}}(A_K \cap A_{\neg\varphi})$. This correponds to teh language containment test $L(K) \subseteq L(\varphi)$, which is equivalent to the modelhood test $K \models \varphi$. The trick is that the complementation of the property, that is required for the language containment test, is pushed off from the complementation of a Büchi automaton to the negation of a temporal logic formula, which is trivial. This approach is used, for example, by the SPIN model checker~\cite{1997_spin}. The disadvantage is that the typically used temporal logic LTL is less expressive than Büchi automata, and hence the breadth of properties that can be expressed is limited. It has been stated that the expressivity of LTL is unsufficient for industrial verification applications~\cite{2007_vardi_model_checking}.

For more information on model checking, as well as other formal verification techniques, we refer to the following works:~\cite{huth2004logic}\cite{ben2012mathematical}\cite{baier2008principles}.

As can be seen from these elaborations, having efficient procedures for Büchi complementation would be of great practical value. Even though handling the ``worst-cases'' will forever be unefficient,


% One of them is to use a deterministic, rather than a non-deterministic, Büchi automaton for representing the property~\cite{1995_tasiran}\cite{2007_vardi_model_checking}. This is because the complementation of determinstic Büchi automata is easy (it can be done in polynomial time and linear space~\cite{Kurshan198759}). This has however the disadvantage that the resulting deterministic automaton may be considerably bigger than an equivalent non-deterministic automaton, and that it is generally more complicated and less intuitive to specify a property as a deterministic automaton~\cite{1995_tasiran}.

%  no verification tools that include the complementation of a non-deterministic property automaton, because of the sheer time and computing resources that it entails.


% As we have seen in the previous section, solving the language containment problem $L(S) \subseteq L(P)$ is done by translating it to the automata-theoretic problem $L(S \cap \cl1{P}) = \varnothing$, which in turn is solved by the following three steps:

% \begin{enumerate}
% \item Construct the \textit{complement} $\cl1{P}$ of the property automaton $P$
% \item Construct the \textit{intersection} automaton, say $A_{S,\cl1{P}}$, of $S$ and $\cl1{P}$
% \item Test $A_{S,\cl1{P}}$ for \textit{emptiness}
% \end{enumerate}

% The formal verification problem is thus reduced to three operations on Büchi automata, \textit{complementation}, \textit{intersection}, and \textit{emptiness testing}. Complementation is clearly the problem child of this triple. For intersection and emptiness testing of Büchi automata there exist efficient solutions~\cite{2007_vardi_model_checking} (cf.~\cite{1996_vardi}). Büchi complementation, however, is so complex that it makes the entire approach impractical~\cite{1995_tasiran}. According to~\cite{2007_vardi_model_checking}, there are so far no verification tools that include the complementation of a non-deterministic property automaton, because of the sheer time and computing resources that it entails.

% Instead, verification tools apply different ways to circumvent the need for complementing non-deterministic property automata. One of them is to use a deterministic, rather than a non-deterministic, Büchi automaton for representing the property~\cite{1995_tasiran}\cite{2007_vardi_model_checking}. This is because the complementation of determinstic Büchi automata is easy (it can be done in polynomial time and linear space~\cite{Kurshan198759}). This has however the disadvantage that the resulting deterministic automaton may be considerably bigger than an equivalent non-deterministic automaton, and that it is generally more complicated and less intuitive to specify a property as a deterministic automaton~\cite{1995_tasiran}.

% Another way to cirvumvent the need for Büchi complementation is to use a slightly different approach to automata-theoretic model checking (in Figure~\ref{model_checking}, a sibling of the language containment approach)~\cite{2007_vardi_model_checking}. In this approach, the property is not specified as a Büchi automaton, but as a linear temporal logic (LTL) formula $\varphi$. The formula $\varphi$ is then negated ($\neg\varphi$) and translated to a Büchi automaton $A_{\neg\varphi}$. If $A_S$ is the system automaton, then the verification step is done by testing $L(A_S \cap A_{\neg\varphi}) = \varnothing$. This works because $A_{\neg\varphi}$ is equivalent to $\cl1{A_{\varphi}}$, that is, the complement of the automaton representing the property. In this way we push off complementation from Büchi automata to LTL formulas, in which case it is trivial. A verification tool that uses this approach is the SPIN model checker~\cite{1997_spin}. The disadvantage of this approach is that LTL is less expressive than Büchi automata, and thus allows to express fewer properties. It has even been stated that the set of properties that can be expressed with LTL is unsufficient for industrial applications~\cite{2007_vardi_model_checking}.

% Summarising, we can say that automata-theoretic model checking is possible without Büchi complementation. However, the language containment approach with the direct specification of the property as a non-deterministic Büchi automaton has important practical advantages. Thus, finding more efficient ways for the complementation of non-deterministic Büchi automata would be of high practical value~\cite{2007_vardi_model_checking}. This motivates the fundamental topic of this thesis. In the next sections, we will see how our work specifically contributes to this quest.


\section{Motivation}
In the previous section we have seen that Büchi complementation is complex, and that it would be of practical value to better understand this complexity. In this section, we highlight the need for looking at this complexity in a way that has not received much attention in the past, namely empirically rather than theoretically.

In the following, we first present the traditional way of analysing the worst-case performance of complementation constructions, and then describe the empirical way for investigating their actual peformance. This includes a review of the work that has been done so far. Note that we are using the terms complexity and performance interchangeably, and they both mean basically state growth.

\subsection{Theoretical Investigation of Worst-Case Performance}
The traditional performance measure for Büchi complementation constructions is their \textit{worst-case state growth}\footnote{As mentioned previously, also known as state complexity, state blow-up, or state explosion.}. This is the maximum number of states the construction \textit{can} generate, in relation to the number of states of the input automaton.

For example, the initial complementation construction by Büchi (1962)~\cite{buchi1960decision} has a worst-case state growth of $2^{2^{O\left(n\right)}}$ does not mean that it produces a larger complement than Schewe's construction, for this concrete example. It might well be smaller. In fact, worst-case state complexities only allow to adequately deduce something about the specific worst-cases, and not about all the other automata. From a practical point of view, these worst cases are however not interesting, as their application is impracticable anyway (at least starting from a certain input automaton size). , where $n$ is the number of states of the input automaton. At this point, two short comments. First, the state growth is often not given as an exact function, but uses the big-O notation. Second, for notating state growths, we will consistenly use the variable $n$, whose meaning is the number of states of the input automaton. This means, for example, that for an input automaton with 8 states, the maximum number of states that the output automaton of Büchi's construction can have is $1.16 \times 10^{77}$ (if assuming the concrete function $2^{2^n}$).

Different constructions exhibit different worst-case state growths, and one of the main objectives of construction creators is to reduce this number. For example, the much more recent construction by Schewe (2009)~\cite{schewe2009buchi} has a worst-case state growth as low as $(0.76n)^n + n^2$. Given an input automaton with 8 states, the maximum number of states of the output automaton is approximately 119.5 million.

A related objective of research is the quest for the theoretical worst-case state growth of Büchi complementation \textit{per se}. A first result of $n!$ has been proposed in 1988 by Michel~\cite{michel1988}. He proved that there exists a family of automata whose complement \textit{cannot} have less than $n!$ states (these automata are known as Michel automata, and we will use them as part of the test data for our experiments). This proves a \textit{lower bound} for the fundamental worst-case complexity of Büchi complementation, as it is not known whether the Michel automata are the real worst cases, or if there are even worse cases. Indeed, in 2007, Yan~\cite{DBLP:journals/corr/abs-0802-1226} proved a new higher lower bound of $(0.76n)^n$ (Michel's $n!$ corresponds to approximately $(0.36n)^n$~\cite{DBLP:journals/corr/abs-0802-1226}). The worst-case state growth of a concrete construction naturally serves as an \textit{upper bound} to a known lower bound. Given Schewe's number $(0.76n)^n n^2$, the lower bound of $(0.76n)^n$ by Yan is regarded as ``sharp'', as the gap between the lower and upper bound is very narrow, and consequently, the lower bound canot rise much anymore.

Many construction developers aim at bringing the worst-case state growth of their construction close to the currently known lower bound. It goes so far that a construction matching this lower bound is regarded as ``optimal''. 


\subsection{Need for Empirical Investigation of Actual Performance}
Worst-case state growths are interesting from a theoretical point of view, but they are poor guides to the actual performance of a construction~\cite{2011_tsai}. For example, if we have a concrete automaton of, say, 15 states, and we complement it with Schewe's construction, the fact that the worst-case state complexity is $(0.76n)^n n^2$ does not reveal anything about how the construction will perform on this concrete automaton. In any case, we are not expecting the complement to have 1.6 quintillion ($1.6 \times 10^{18}$) states (which would be the worst case), because this would most likely be practically infeasible.

Furthermore, if a construction has a higher worst-case state growth than another, it does not mean that it performs worse on a concrete case. In fact, worst-case state complexities only allow to adequately deduce the performance on the worst-case automata, but not on all the other automata. However, from a practical point of view, these worst cases are not interesting, as their application in practice is anyway infeasible~\cite{1995_tasiran} (at least starting from a certain input automaton size).

From a practical perspective we are interested how constructions perform on automata as they could occur in a concrete application of Büchi complementation, such as automata-theoretic model checking. This may include questions like the following. What is a reasonable complement size to expect for the given automaton with $n$ states? Are there generally easier and harder automata? What are the factors that make an automaton especially easy or hard? How does the performance of different constructions on the same automata vary? Are there constructions which are better suited for a certain type of automata than other constructions?

Questions like this can be attempted to answer by empirical peformance investigations. As its two most important elements this includes an \textit{implementation} of the investigated constructions and \textit{test data}. With test data, we mean a set of concrete automata on which the implementations of the constructions are run. The analysis is then done on the generated complement automata.

There have been relatively few empirical attempts in the history of Büchi complementation~\cite{2011_tsai}, compared to the number of theoretical works. In the following, we give  (non-exhaustive) list of empirical works in the past that illustrate the approach, and also show the line of research in which the work of this thesis is situated.

{\setlist[description]{leftmargin=0.5cm, itemsep=\parskip}
\begin{description}
\item[1995] Tasiran et al.~\cite{1995_tasiran} create an efficient implementation of Safra' construction\cite{1988_safra_2} (determinisation-based) and used it for for automata-theoretic model checking tasks with the HSIS verification tool~\cite{1994_hsis}. They state that they could easily complement property automata with some hundreds of states, however, they do not provide a statistical evaluation of the results.

\item[2003] Gurumurthy et al.~\cite{2003_Gurumurthy} implement Friedgut, Kupferman, and Vardi's construction~\cite{Kupferman:2001} (rank-based) along with various optimisations that they propose as a part of the tool Wring~\cite{somenzi2000efficient}. They complement 1000 small automata, generated by translation from LTL formulas, and evaluate exeuction time, and number of states and transitions of the complement for the different versions of the construction.

\item[2006] Althoff et al.~\cite{2006_althoff} implement Safra's~\cite{1988_safra_2} and Muller and Schupp's~\cite{Muller199569} determinisation constructions\footnote{These determinisation constructions transform a non-deterministic Büchi automaton to a deterministic Rabin automataon (see Section~\ref{om-automata}), however, the are used as the base for determinisation-based complementation constructions.} in a tool called OmegaDet, applied them on the Michel automata with 2 to 6 states, and compared the number of states of the determinised output automata.

\item[2008] Tsay et al.~\cite{2008_goal_ext} carry out a first comparative experiment with the publicly available\footnote{\url{http://goal.im.ntu.edu.tw/wiki/doku.php}} \goal{} tool~\cite{2007_goal}\cite{2008_goal_ext}\cite{2009_goal}\cite{2013_goal}. They include the constructions by Safra~\cite{1988_safra_2} (determinisation-based), Piterman~\cite{2007_piterman} (determinisation-based), Thomas~\cite{1999_thomas} (WAPA\footnote{Via \textbf{W}eak \textbf{A}lternating \textbf{P}arity \textbf{A}utomaton}), and Kupferman and Vardi\cite{Kupferman:2001} (rank-based or WAA\footnote{Via \textbf{W}eak \textbf{A}lternating \textbf{A}utomaton}). These constructions are pre-implemented in \goal. As the test data, they use 300 Büchi automata, translated from LTL formulas, with an average size of 5.4 states. They evaluate and compare execution times, as well as number of states and transitions of the complements.

\item[2009] Kamarkar and Chakraborty~\cite{2009_karmarkar} propose an improvement of Schewe's construction~\cite{schewe2009buchi} (rank-based) and implement it, as well as Schewe's original construction, on top of the model checker NuSMV~\cite{1999_nusmv}\cite{2002_nusmv}. They run the constructions on 12 test automata and compare the sizes of the complements. Furthermore, they run the same tests with the constructions by  Kupferman, and Vardi~\cite{Kupferman:2001} (rank-based or WAA) and Piterman~\cite{2007_piterman} (determinisation-based) within \goal, and compare the results to the ones of their implementation of Schewe's construction.

\item[2010] Tsai et al.~\cite{2011_tsai} (paper entitled ``State of Büchi Compelentation'') carry out another experiment with \goal. They compare the constructions by Piterman~\cite{2007_piterman} (determinisation-based), Schewe~\cite{schewe2009buchi} (rank-based), and Vardi and Wilke~\cite{vardi2007automata} (slice-based), with various optimsiations that they propose in the same paper. As the test data, they use 11,000 randomly generated automata with 15 states and an alphabet size of 2. The test set is organised into 110 automata classes that consist of the combinations of 11 transition densities and 10 accceptance densities. This test set is repeatedly used in subsequent work (including in this thesis), and we will refer to it as the \goal{} test set (because it has been generated with the \goal{} tool). Tsai et al. provide sophisticated evaluation of the states of the complements for all the tested constructions and construction versions.

\item[2010] Breuers~\cite{2010_breuers_bsc} proposes an improvement for the construction by Sistla, Vardi, and Wolper~\cite{PrasadSistla1987217} (Ramsey-based), and creates an implementation of it. He generates his own test data (inspired by the work of Tsai et al.~\cite{2011_tsai}) consisting of \textit{easy}, \textit{medium}, and \textit{hard} automata, based on different transition density and acceptance density values. He evaluates the complement sizes produced by the construction for autoamta of sizes 5, 10, and 15 of all these difficulty categories.

\item[2012] Breuers et al.~\cite{2012_breuers} wrap the implementation of their improvement of Sistla, Vardi, and Wolper's construction~\cite{PrasadSistla1987217} in the publicly available tool Alekto\footnote{\url{http://www.automata.rwth-aachen.de/research/Alekto/}}, and and run it on the \goal{} test set. They compare the generated complement sizes, as well as the number of aborted complementation tasks (due to exceeding resource requirements) to the corresponding result for different constructions on the same test set by Tsai et al.~\cite{2011_tsai}.

\item[2013] Göttel~\cite{2013_bsc_goettel} creates a C implementation of the Fribourg construction~\cite{2014_joel_ulrich}, including the R2C optimisation (see Chapter~\ref{chap_construction}), and executes it on the \goal{} test set, as well as on the Michel automata with 3 to 6 states. He analyses the resulting complement sizes and execution times separately for each of the 110 classes that the \goal{} test set consists of. The Fribourg construction\footnote{The authors of the constructions use the name \textit{subset-tuple construction} (see~\cite{2014_joel_ulrich}), however, in this thesis, we will use the name \textit{Fribourg construction}.} is a slice-based complementation construction that is being developed at the university of Fribourg, and which lies at the heart of this thesis. The entire Chapter~\ref{chap_construction} of this thesis is dedicated to explaining the Fribourg construction.
\end{description}}


\section{Aim and Scope}
The aim of this thesis is an in-depth empirical performance investigation of the Fribourg construction. As mentioned, the Fribourg construction is a Büchi complementation construction that is being developed at the University of Fribourg~\cite{2014_joel_ulrich}. By empirically investigating the behaviour of this specific construction, we want to follow up the track of empirical research that we have outlined in the last section.

This thesis is certainly not sufficient to describe the performance of the Fribourg construction in its entiretey, or in a way that is adequate to be relied on in industiral applications. Neither this thesis can answer general questions about the observed behaviour of Büchi complementation. Rather, we see this piece of work as a mosaic stone that we add to the very complex and multi-faceted picture of the complexity of Büchi complementation.

The empirical performance investigation will include testing of different versions of the construction, and comparison with other complementation constructions...


Aim: empirical performance investigation of a specific Büchi complementaiton construction, comparison with other constructions

Scope: two test sets, relatively small automata, no real world or ``typical'' examples,

% Significance: advance knowledge, contribute to solutin of practical problem, novel use of a procedure or technique?

\section{Overview}


% \subsubsection{Worst-Case State Growth as Main Performance Measure} 
% Since the introduction of Büchi automata in 1962, many different Büchi complementation constructions have been proposed (see our review in Section~\ref{2_constructions}). The main performance measure for these constructions has usually been their so-called \textit{worst-case state growth} or \textit{worst-case state complexity} (in the following, we will use these two terms interchangeably\footnote{Further terms used in the literature are state blow-up or state explosion.}).

% State growth basically denotes the number of states of the output automaton in relation to the number of states of the input automaton to a complementation construction. Each automaton has thus its specific own state growth with each construction. The worst-case state growth of a construction results from a theoretical worst-case automaton, which has a higher state growth than any other automaton. In different words, the worst-case state growth is the maximum number of states that a concstruction \textit{can} generate.

% The state growth and the resulting time and space complexity is the biggest issue of Büchi complementation. The worst-case state growth seems to ``distill'' this issue to a single number which is practical to use as a concise performance measure and to compare different constructions with each other. Thus, much of the research effort in Büchi complementation construction has gone into reducing the worst-case state growth.

% For example, the complementation construction that has been described in 1962 by Büchi himself~\cite{buchi1960decision} has a doubly exponential worst-case state growth of $2^{2^{O\left(n\right)}}$, where $n$ is the number of states of the input automaton\footnote{In the following, we will always notate state growths as a function of $n$, and $n$ will always be the number of states of the input automaton.}. Note that worst-case state growths are often not given as exact functions, but include the big O notation. A later construction from 1987 by Sistla, Vardi, and Wolper~\cite{PrasadSistla1987217} reduces this worst-case state complexity to a singly exponential funtion of $O\left(2^{4n^2}\right)$. More recently, a construction from 2006 by Friedgut, Kupferman and Vardi from 2006~\cite{friedgut2006buchi} has a worst-caste state complexity of only $(0.96n)^n$.

% In parallel to the quest for complementation constructions with a low worst-case state complexity, there is a quest for finding the worst-caste state complexity of Büchi complementation itself. This is done by showing a theoretical minimum state growth of certain automata which even an ideal complementation construction could not undermatch. In this way, one proves a \textit{lower bound} for the worst-case complexity of Büchi complementation (it is still possible that there exist automata with an even higher theoretical minimum state growth). In 1988 Michel proved  such a lower bound of $n!$~\cite{michel1988} (in a different notation approximately $(0.36n)^n$). In 2008, Yan proved a new lower bound of $(0.76n)^n$~\cite{DBLP:journals/corr/abs-0802-1226}. This result is still valid at the time of this writing.

% A lower bound of $(0.76n)^n$ means that no complementation construction can ever have a worst-case state growth lower than $(0.76n)^n$. Consequently, a construction that achieves this worst-case state growth is commonly regarded as ``optimal''. 

% \subsubsection{Importance of Empirical Performance Investigations}
% Worst-case state growths are certainly interesting from a theoretical point of view. However, they can be seen as only one aspect of the performance of Büchi complementation constructions. The reality is certainly much more complex. If, for example, one is interested in practical questions (like how does a construction peform on a concrete automaton?), then worst-case state growths are a poor guide to the concrete performance of the construction~\cite{2011_tsai}.

% For example, if we have a concrete automaton, then the worst-case state growth does not reveal anything about how big the complement of this concrete automaton will be. It is not even clear whether in a concrete case a construction with a lower worst-case state growth produces a smaller complement than a construction with a higher worst-case state growth.

% Worst-case state growths are mainly meaningful for the worst cases. However, if these worst cases would occur, then all constructions would be far from practical anyway~\cite{1995_tasiran}. If we take for example Friedgut, Kupferman, and Vardi's construction~\cite{friedgut2006buchi} with a worst-case complexity of $(0.96n)^n$, Sistla, Vardi, and Wolper's construction with a worst-case complexity of $O\left(2^{4n^2}\right)$, and a ``worst-case'' automata of, say, 15 states, then the complements of the two constructions would have $2.73 \times 10^17$ (23.7 million billion) and $8.45 \times 10^270$ states, respectively. While this difference is still huge, in practice this does not matter much, because both cases are most likely anyway not feasible in practice.

% Hence, with regard to the practical application of Büch complementation (see Section~\ref{1_context}), we believe that it is important to investigate the \textit{actual} performance of Büchi complementation constructions on \textit{concrete} input automata. This is done by \textit{empirical} performance investigations that includes an implementation of the investigated construction, and a set of test data.







% 2. The Büchi complementation problem
% 2.1 Preliminaries
% Used naming convention (NBW, DBW, etc.)
%   2.1.1 Büchi automata
%     - Definition
%     - DBW vs. NBW
%     - Equivalence with omega-regular languages (NBW)
%     - DBW weaker than NBW (proof)
%   2.1.2 Other omega-automata
%     - Muller, Rabin, Streett, Parity
%     - McNaughton's Theorem (NBW = DMW)
%     - Complete picture of equivalences
%   2.1.3 Complementation of Büchi automata
%     - NBW closed under complementation (DBW not, proof)
%     - Example NFA/DFA
%     - Complementation of DBW (Kurshan)
%   2.1.4 Complexity of Büchi Complementation
%     - Notion of state blow-up
%     - Lower bounds for the state blow-up
% 2.2 Review of Büchi Complementation Constructions
% 2.3 Empirical Performance Investigations