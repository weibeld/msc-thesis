
In this chapter we come to the core of this thesis, namely to test how the Fribourg construction performs on real test automata. We are interested in two things. First, how the different versions of the Fribourg construction compare to each other. That is, with which optimisations the Fribourg construction is most efficient. Second, we compare the Fribourg construction to other complementation constructions. We can refer ot the first type of investigations as the \textit{internal} tests, and to the second one as the \textit{external} tests.

To do these investigations, we implemented the Fribourg construction as a plugin for an \om-automata manipulation tool called GOAL. GOAL already contains implementations of the most important Büchi complementation constructions. That is, with our plugin, the Fribourg construction lives next to these other constructions in the GOAL tool, and can be easily compared to them. With the plugin in place, we then performed the actual internal and external tests. To do so, we defined a set of test data consisting of totally 22.000 automata. The performance investigation consists then in basically complementing each of these automata with the different constructions, and comparing the results. Our main performance metric is the number of generated states for the output automata. The computations, which due to the complexity of Büchi complementation are quite heavy, were executed on a high-performance computing cluster at the University of Bern, Switzerland.

In this chapter, we are going to describe each of these points, including our concrete experiment setup. The results of the tests are presented and discussed in Section~\ref{results}.

\section{Implementation}
\subsection{GOAL}
\label{goal}
GOAL stands for Graphical Tool for Omega-Automata and Logics and has been developed at the National University of Taiwan since 2007~\cite{2007_goal,2008_goal_ext}. The tool is based on the three pillars, \om-automata, temporal logic formulas, and games. It allows to create instances of each of these types, and manipulate them in a multitude of ways. Relevant for our purposes are the \om-automata capabilities of GOAL.

With GOAL, one can create Büchi, Muller, Rabin, Streett, parity, generalised Büchi, and co-Büchi automata, either by manually defining them, or by having them randomly generated. It is then possible to perform a plethora of operations on these automata. The entirety of provided operations are too many to list, but they include containment testing, equivalence testing, minimisation, determinisation, conversions to other \om-automata types, product, intersection, and, of course, complementation.

All this is accessible by both, a graphical and a command line interface. The graphical interface is shown in Figure~\ref{goal_gui}. Automata are displayed in the main editor window of the GUI. They can be freely edited, such as adding or removing states and transitions, and arranging the layout. There are also various layout algorithms for automatically laying out large automata. Most of the functionality provided by the graphical interface is also accessible via a command line mode. This makes it suitable for automating the execution of operations.

\begin{figure}
\begin{center}
\includegraphics[scale=0.5]{figures/goal_gui.png}
\caption{Graphical interface of GOAL.}
\label{goal_gui}
\end{center}
\end{figure} 

For storing automata, GOAL defines an own XML-based file format, called GOAL File Format, usually indicated by the file extension gff.

An important design concept of GOAL is modularity. GOAL uses the Java Plugin Framework (JPF)~\footnote{http://jpf.sourceforge.net/}, a library for building modular and extensible Java applications. A JPF application defines so-called extension points for which extensions are provided. These extensions contain the actual functionality of the application. Extensions and extension points are bundled in plugins, the main building block of a JPF application. It is therefore possible to extend an existing JPF application by bundling a couple of new extensions for existing extensions points in a new plugin, and installing this plugin into the existing application. On the next start of the application, the new functionality will be included, all without requiring to recompile the existing application or to even have its source code.

GOAL provides a couple of extensions points, such as \textit{Codec}, \textit{Layout}, or \textit{Complementation Construction}. An extension for \textit{Codec}, for example, allows to add the handling of a new file format which GOAL can read from and write to. With an extension for \textit{Layout} one can add a new layout algorithm for laying out automata in the graphical interface. And an extension to \textsf{Complementation Construction} allows to add a new complementation construction to GOAL. This is how we added the Fribourg construction to GOAL, as we will further explain in Section~\ref{implementation}.

There are a couple of Büchi complementation constructions pre-implemented in GOAL. Table~\ref{goal_constructions} summarises them, showing for each one its name on the graphical interface and in the command line mode, and the reference to the paper introducing it. As can be seen, the most important representants of all the four approaches (Ramsey-based, determinisation-based, rank-based, and slice-based, see Chapter~\ref{background}) are present. In addition to the listed constructions, GOAL also contains Kurshan's construction and classic complementation. These are for complementing DBW and NFA/DFA, respectively, and thus not relevant to us.

\begin{table}
\caption{The complementation constructions implemented in GOAL (version 2014-11-17).}
\begin{center}
\begin{tabular}{|l|l|l|}
\hline
Name & Command line & Reference \\
\hline
Ramsey-based construction & ramsey & Sistla, Vardi, Wolper (1987)~\cite{PrasadSistla1987217} \\
\hline
Safra's construction & safra & Safra (1988)~\cite{1988_safra_1} \\
\hline
Modified Safra's construction & modfiedsafra & Althoff (2006)~\cite{2006_althoff} \\
\hline
Muller-Schupp construction & ms & Muller, Schupp (1995)~\cite{Muller199569} \\
\hline
Safra-Piterman construction & piterman & Piterman (2007)~\cite{2007_piterman} \\
\hline
Via weak alternating parity automaton & wapa & Thomas (1999)~\cite{1999_thomas} \\
\hline
Via weak alternating automaton & waa & Kupferman, Vardi (2001) \cite{Kupferman:2001} \\
\hline
Rank-based construction & rank & Schewe (2009) \cite{schewe2009buchi} \\
\hline
Slice-based construction (preliminary) & slice -p & Vardi, Wilke (2007) \cite{vardi2007automata} \\
\hline
Slice-based construction & slice & Kähler, Wilke (2008) \cite{2008_kaehler} \\
\hline
\end{tabular}
\end{center}
\label{goal_constructions}
\end{table}

One of the constructions can be set as the default complementation construction. It is then possible to invoke this construction with the shortcut Ctrl-Alt-C. Furthermore, the default complementation constructions will be used for the containment and equivalence operations on Büchi automata, as they include complementation.

Complementation constructions in GOAL can define a set of options that can be set by the user. In the graphical interface this is done at the start of the operations via a dialog window, in the command line mode the options are specified as command line arguments. Figure~\ref{goal_complementation_options} shows the options dialog of the Safra-Piterman construction. Complementation options allow to play with different configurations and variants of a construction, and we will make use of them for including the optimisations presented in Chapter~\ref{fribourg_construction} to our implementation of the Fribourg construction.


\begin{figure}
\begin{center}
\includegraphics[scale=0.5]{figures/goal_complementation_options.png}
\caption{Complementation constructions in GOAL can have a set user-selectable options. Here the options of the Safra-Piterman construction.}
\label{goal_complementation_options}
\end{center}
\end{figure}

For most complementation constructions (all listed in Table~\ref{goal_constructions} except the Ramsey-based construction) there is also a version for step-by-step execution. In this case, the constructions define so-called steps and stages, through which the user can iterate independently. This is a great way for understanding how a complementation construction works, and for investigating specific cases in order to potentially further improve the construction. 


\subsection{Implementation of the Construction}
\label{implementation}
We implemented the Fribourg construction, including its optimisations, in Java as a plugin for GOAL. This means that after installing out plugin to an existing GOAL installation\footnote{As the plugin interfaces of GOAL have recently changed, the can be used only for GOAL versions 2014-11-17 and newer.}, the Fribourg construction will be an integral part of GOAL and can be used in the same way as any other pre-existing complementation construction.

To keep the Fribourg construction flexible, we made use of options. The three optimisations described in Section~\ref{optimisations} are presented to the user as selectable options. Additionally, we included several further options. Table~\ref{goal_fribourg_options} lists them all. For convenience, we use for each options a short code name, which is also used as the option name in the command line mode.

\begin{table}
\caption{The options for the Fribourg construction.}
\begin{center}
\begin{tabular}{|l|l|}
\hline
Code & Description \\ \hline
m1 & Component merging optimisation \\ \hline
m2 & Single 2-coloured component optimisation \\ \hline
r2c & Deleting states with rightmost colour 2, if automaton is complete \\ \hline
c & Make input automaton complete \\ \hline
macc & Maximise accepting states of input automaton \\ \hline
r & Remove unreachable and dead states from output automaton \\ \hline
rr & Remove unreachable and dead states from input automaton \\ \hline
b & Use the ``bracket notation'' for state labels \\ \hline
\end{tabular}
\end{center}
\label{goal_fribourg_options}
\end{table}

The first three items in Table~\ref{goal_fribourg_options}, m1, m2, and r2c, correspond to the optimisations M1, M2, and R2C, described in Section~\ref{optimisations}. As the M2 optimisation requires M1, our implementation makes sure that the m2 option can only be selected if also the m1 option is selected. The \tec{c} option, for making the input automaton complete before starting the actual construction, is intended to be used with the \tec{r2c} option. In this way, the R2C optimisation can be forced to apply. This idea results from previous work that investigated whether making the input automaton complete plus the application of the R2C optimisation brings an improvement over the bare Fribourg constructoin~\cite{2013_bsc_goettel}. The result was negative, that is, the construction performs worse with this variant on practical cases. Also note that using the \tec{c} option alone, very likely decreases the performance of the construction, because the automaton is made bigger if it is not complete.

The \tec{macc} and \tec{r} options are common among the other complementation constructions in GOAL. The first one, \tec{macc}, maximises the accepting set of the input automaton. That means, it makes as many states accepting as possible without changing the automaton's language. This should help to make the complement automaton smaller. The \tec{r} options prunes unreachable and dead states from the complement automaton. Unreachable states are states that cannot be reached from the initial states, and dead states are states from where no accepting state can be reached. Clearly, all the runs containing an unreachable or dead state are not accepting, and thus these states can be removed from the automaton without changing its language. The complement automaton can in this way be made smaller. The \tec{rr} option in turn removes the unreachable and dead states from the \emph{input} automaton. That is, it makes the input automaton smaller, before the actual construction starts, what theoretically results in smaller complement automaton.

Finally, the \tec{b} option affects just the display of the state labels of the complement automaton. It uses an alternative notation which uses different kinds of brackets, instead of the explicit colour number, to indicate the colours of sets. In particular, 2-coloured sets are indicated by square brackets, 1-coloured sets by round parenthesis, and 0-coloured sets by curly braces. Sets of states of the upper part of the automaton are enclosed by circumflexes. This notation, although being very informal, has proven to be very convenient during the development of the construction.

When we developed the plugin, we aimed for a complete as possible integration with GOAL. We integrated the Friboug construction in the graphical, as well as in the command line interface. We added a step-by-step execution of the construction in the graphical interface. We provided that customised option configurations can be persistently saved, and reset to the defaults at any time. We also integrated the Fribourg construction in the GOAL preferences menu so that it can be selected as the default complementation construction. In this way, it can be invoked with a key-shortcut and it will also be used for the containment and equivalence operations. Our goal is that once the plugin is installed, the Fribourg construction is as seamlessly integrated in GOAL as all the other pre-existing complementation construction.

The complete integration allows us to publish the plugin so that it can be used by other GOAL users. At the time of this writing, the plugin is accessible at \url{http://goal.s3.amazonaws.com/Fribourg.tar.gz} and also over the GOAL website\footnote{http://goal.im.ntu.edu.tw/}. The installation is done by simply extracting the archive file and copying the contained folder to the \tec{plugins/} folder in the GOAL system tree. No compilation is necessary. The same plugin and the same installation procedure works FOR Linux, Mac OS X, Microsoft Windows, and other operating systems that run GOAL.

Since between the 2014-08-08 and 2014-11-17 releases of GOAL certain parts of the plugin interfaces have changed, and we adapted our plugin accordingly, the currently maintained version of the plugin works only with GOAL versions 2014-11-17 or newer. It is thus essential for any GOAL user to update to this version in order to use our plugin.

\subsection{Verification of the Implementation}

See UBELIX/jobs/2014-11-25

Can we do a complement-equivalence test with all the 11,000 automata of size 15 in the test set?

Of course it is needed to test whether our implementation produces correct results. That is, are the output automata really the complements of the input automata? We chose doing so with an empirical approach, taking one of the pre-existing complementation constructions in GOAL as the ``ground truth''. We can then perform what we call complementation-equivalence tests. We take a random Büchi automaton and complement it with the ground-truth construction. We then complement the same automaton with our implementation of the Fribourg construction, and check whether the two complement automata are equivalent. Provided that the ground-truth construction is correct, we can show in this way that our construction is correct for this specific case.

We performed complementation-equivalence tests for the Fribourg construction with different option combinations. In particular, we tested the configurations \tec{m1}, \tec{m1}+\tec{m2}, \tec{c}+\tec{r2c}, \tec{macc}, \tec{r}, \tec{rr}, and the construction without any options. For each configuration we tested 1000 random automata of size 4 and with an alphabet of size 2 to 4. As the ground-truth construction we chose the Safra-Piterman construction. In all cases the complement of the Fribourg construction was equivalent to the complement of the Safra-Piterman construction.

Doubtlessly, it would be desirable to test more, and especially bigger and more diverse automata. However, by doing so one would quickly face practical problems due to long complementation times with bigger automata and larger alphabets, and high memory usage. For our current purpose, however, the tests we did are enough for us to be confident that our implementation is correct.

\section{Test Data}
The core of this thesis is to empirically investigate the performance of the Fribourg construction. To this end, we ``feed'' a set of concrete automata to the Fribourg construction and analyse the results. This set of automata is the test data and is of crucial importance for the validity of the performance investigation. Test data should not be biased in favour or unfavour of one of the tested algorithms. Ideally, publicly available, or even standardised, test data is used, that has been previously used in other experiments. Furthermore, test data should cover interesting cases that reveal important aspects about the algorithm.

For our investigation we chose two test sets that are publicly known or available and cover very different scenarios. While one of them consists of 11,000 automata and tries to cover a broad range of ``everyday'' complementation problems, the other one consists of just 4 automata, that are however very special. We call the former one the \goal{} test set and the latter on the Michel test set, and we are going to describe them in the following two sections.


\subsection{GOAL Test Set}
\label{4_goal_testset}
The \goal{} test set has been created by Tsai~et~al. for the experiments described in the the paper \textit{State of Büchi Complementation}~\cite{2010_tsai}. In this paper, the authors carry out empirical performance investigations of Büchi complementation constructions that are very similar to our own investigations. The experiments were run with the \goal{} tool, hence the name \goal{} test set. The idea leading the design of this test set was to generate a large set of complementation problems with various difficulty levels.

The test set consists of 11,000 automata of size 15, and with an alphabet size of 2. It is structured into 110 classes that are combinations of two parameters, the so called transition density and the acceptance density. Simply speaking, the transition density determines the number of transitions in an automaton, and the acceptance density determines the number of accepting states.

More precisely, the transition density $t$ is defined as follows. Let $n$ be the number of states of automaton $A$, and $t$ its transition density. Then $A$ contains $tn$ transitions for each symbol of the alphabet\footnote{In the case $tn$ is not an integer, it is rounded up to the next integer.}. That is, if one of our test set automata with 15 states and the alphabet ${0, 1}$ has a transition density of 2, then it contains exactly 30 transtions for symbol $0$ and 30 transitions for symbol $1$. Consequently, each state has on average two outgoing and two incoming transitions for each symbol. Therefore, the transition density can also be seen as the average number of outgoing and incoming transitons per alphabet symbol that a state has.

The acceptance density $a$ is defined as the ratio of accepting states to non-accepting states in the automaton. That is, if automaton $A$ has $n$ states and an acceptance density of $a$, then it contains $an$ accepting states\footnote{Again, if $an$ is not an integer, it is rounded up to the next integer.}. The acceptance density is thus bound to a number between 0 and 1. For example, if an automaton of our test set has an acceptance density of 0.1, then it has 2 accepting states, and if an automaton has an accepting density of 1, then all of its states are accepting states.

In the \goal{} test set there are 11 transiton densities and 10 acceptance densities. The transition densities range from 1 to 3 in steps of 0.2, and the acceptance densities range from 0.1 to 1 in steps of 0.1. Thus, we have
\begin{align*}
t^\prime& = \{ 1.0,\,1.2,\,1.4,\,1.6,\,1.8,\,2.0,\,2.2,\,2.4,\,2.6,\,2.8,\,3.0 \} \\
a^\prime & = \{ 0.1,\,0.2,\,0.3,\,0.4,\,0.5,\,0.6,\,0.7,\,0.8,\,0.9,\,1.0 \}
\end{align*}

The 110 classes result from the Cartesian product of $t^\prime$ and $a^\prime$. Each class consequently contains 100 automata. In this thesis, we will often present these classes as a $11 \times 10$ matrix where the rows represent the transition densities $t^\prime$, and the columns represent the acceptance densities $a^\prime$. Within the given constraints, the 11,000 automata were generated at random.

The original \goal{} test set includes another set of 11,000 automata with the same specifications as the first one, except that the size of the automata is 20 instead of 15. In their paper~\cite{2010_tsai}, Tsai~et~al. repeated their experiments with this size 20 test set. In our own investigations we will however only use the size 15 test set. On the one hand, we think that repeating the experiments with the size-20 test set would add only a limited benefit to the insight already gained from the size-15 test set. It might be more beneficial to do further tests with a more diverse type of data (like the Michel test set that we describe in the next section). On the other hand, the size-20 test set may cause practical problems in terms of required computing and time resources. We restrict ourselves therefore to the size-15 test set, and when we refer to the \goal{} test set, we strictly mean the size-15 test set throughout the rest of this thesis.

The \goal{} test set is publicly available from the \goal{} website via the following link: \url{http://goal.im.ntu.edu.tw/wiki/lib/exe/fetch.php?media=goal:ciaa2010_automata.tar.gz}.

We tested each of the 11,000 automata for completeness and universality. As GOAL provides no commands for testing completeness and universality, we created the additional GOAL plugin \textsf{ch.unifr.goal.util} that implements these two operations and makes them accessible over the command line interface. The results of our tests are the following.

\begin{itemize}
\item 990 of the 11,000 automata are complete (9\%)
\item 6,796 of the 11,000 automata are universal (61.8\%)
\end{itemize}

We furthermore analysed how these two properties are distributed over the 110 classes of transition/acceptance density combinations. The results follow below in table form and as a three-dimensional visualisation.

\begin{figure}
  \centering
  \hfill
  \renewcommand{\tabcolsep}{0.1cm}
  \begin{subtable}[t]{0.45\textwidth}
    \centering
    % latex table generated in R 3.1.2 by xtable 1.7-4 package
% Sat Jun  6 16:42:17 2015
\begin{tabular}{|rrrrrrrrrr}
  \hline
0.1 & 0.2 & 0.3 & 0.4 & 0.5 & 0.6 & 0.7 & 0.8 & 0.9 & 1.0 \\ 
  \hline
0 & 0 & 0 & 0 & 0 & 0 & 0 & 0 & 0 & 0 \\ 
  0 & 0 & 0 & 0 & 0 & 0 & 0 & 0 & 0 & 0 \\ 
  0 & 0 & 0 & 0 & 0 & 0 & 0 & 0 & 0 & 0 \\ 
  0 & 0 & 0 & 0 & 0 & 0 & 0 & 0 & 0 & 0 \\ 
  1 & 1 & 0 & 0 & 0 & 1 & 1 & 1 & 0 & 0 \\ 
  0 & 5 & 1 & 2 & 2 & 3 & 1 & 2 & 2 & 2 \\ 
  5 & 10 & 8 & 5 & 3 & 5 & 8 & 6 & 7 & 1 \\ 
  10 & 6 & 11 & 11 & 8 & 6 & 10 & 20 & 9 & 7 \\ 
  17 & 17 & 12 & 16 & 14 & 19 & 22 & 21 & 19 & 19 \\ 
  27 & 20 & 29 & 32 & 26 & 27 & 30 & 25 & 24 & 19 \\ 
  37 & 37 & 40 & 39 & 34 & 37 & 38 & 35 & 38 & 39 \\ 
   \hline
\end{tabular}

    \caption{Completeness}
  \end{subtable}
  \hfill
  \begin{subtable}[t]{0.45\textwidth}
    \centering
    % latex table generated in R 3.1.2 by xtable 1.7-4 package
% Sat Jun  6 16:42:17 2015
\begin{tabular}{|rrrrrrrrrr}
  \hline
0.1 & 0.2 & 0.3 & 0.4 & 0.5 & 0.6 & 0.7 & 0.8 & 0.9 & 1.0 \\ 
  \hline
0 & 0 & 0 & 0 & 0 & 0 & 0 & 0 & 0 & 0 \\ 
  0 & 0 & 0 & 0 & 0 & 0 & 0 & 0 & 0 & 0 \\ 
  0 & 0 & 0 & 0 & 0 & 0 & 0 & 0 & 0 & 0 \\ 
  0 & 0 & 0 & 0 & 0 & 0 & 0 & 0 & 0 & 0 \\ 
  1 & 1 & 0 & 0 & 0 & 1 & 1 & 1 & 0 & 0 \\ 
  0 & 5 & 1 & 2 & 2 & 3 & 1 & 2 & 2 & 2 \\ 
  5 & 10 & 8 & 5 & 3 & 5 & 8 & 6 & 7 & 1 \\ 
  10 & 6 & 11 & 11 & 8 & 6 & 10 & 20 & 9 & 7 \\ 
  17 & 17 & 12 & 16 & 14 & 19 & 22 & 21 & 19 & 19 \\ 
  27 & 20 & 29 & 32 & 26 & 27 & 30 & 25 & 24 & 19 \\ 
  37 & 37 & 40 & 39 & 34 & 37 & 38 & 35 & 38 & 39 \\ 
   \hline
\end{tabular}

    \caption{Universality}
  \end{subtable}
  \renewcommand{\tabcolsep}{0.2cm}
  \hfill

  \hfill
  \begin{subfigure}[t]{0.475\textwidth}
    \centering
    \includegraphics[width=\textwidth]{figures/r/completeness/persp.pdf}
    \caption{Completeness}
  \end{subfigure}
  \hfill
  \begin{subfigure}[t]{0.475\textwidth}
    \centering
    \includegraphics[width=\textwidth]{figures/r/universality/persp.pdf}
    \caption{Universality}
  \end{subfigure}
  \hfill
\caption{Completeness and universality in the GOAL test set.}
\label{4_compl_univ}
\end{figure}

We also tested the automata for emptiness, that is, the contrary of universality. From the 11,000 automata, only 63 are empty, which is a percentage of 0.57\%. As one could expect, most of them are concentrated in the classes with low transition and acceptance densities. For example, 44 of these empty automata are in the classes with a transition density of 1.0.


\subsection{Michel Test Set}
Besides the test set of the GOAL automata, we did all the tests also on the first four Michel automata:
\begin{itemize}
\item Michel N1: 3 states, 7 transitions, 1 accepting state
\item Michel N2: 4 states, 14 transitions, 1 accepting state
\item Michel N3: 5 states, 23 transitions, 1 accepting state
\item Michel N4: 6 states, 34 transitions, 1 accepting state
\end{itemize}

These are the Michel automata that can be processes with our implementation and on our execution environment. Complementing Michel automata N5 and above would already by far exceed our available computing and time resources.


\section{Experimental Setup}

\subsection{Internal Tests}
\label{4_internal}
In the internal tests our aim is to compare different versions of the Fribourg construction. A specific version of the Fribourg construction is composed of a combination of options. Options can be the 

As presented in Section~\ref{optimisations}, there are three optimisations to the Fribourg construction:
\begin{itemize}
\item R2C: if the input automaton is complete, remove all states whose rightmost colour is 2
\item M1: merge certain adjacent sets within a state
\item M2: reduce 2-coloured sets (requires M1)
\end{itemize}

Furthermore, our GOAL plugin includes, among others, the following options:
\begin{itemize}
\item C: make the input automaton complete (by adding a sink state)
\item R: remove unreachable and dead states from the output automaton
\end{itemize}

The versions of the Fribourg construction that we chose for our internal tests consist of combinations of these five options. According to the nature of the two test sets (the GOAL test set and the Michel automata), we chose two different sets of versions for the two test sets. We are now first going to describe the setup for the GOAL test set and then the one for the Michel test set.


\subsubsection{GOAL Test Set}

For the tests on the GOAL test set, we chose the following eight versions of the Fribourg construction:
\begin{enumerate}
\item Fribourg
\item Fribourg+R2C
\item Fribourg+R2C+C
\item Fribourg+M1
\item Fribourg+M1+M2
\item Fribourg+M1+R2C
\item Fribourg+M1+R2C+C
\item Fribourg+R
\end{enumerate}

The first version is the plain Fribourg construction without any optimisations or options. The next two versions are devoted to investigate the R2C optimisation. Version 2 applies the optimsation only to the automata which happen to be complete (and as we have seen in Section~\ref{goal_testset} these are 9\%). Version 3, on the other hand, makes all input automata preliminarily complete by adding a sink state, so that the R2C optimisation can be applied to \textit{all} automata. The question here is, does it pay off to increase the size of the automata by one (for adding the sink state) but then being able to apply R2C, or is it better to not add the extra state but then not applying R2C neither?

Similar investigations about the R2C optimisation of the Fribourg construction have been made by Göttel~\cite{2013_bsc_goettel}. In particular, he compared Version 1 in our above listing with Version 3. His results were that the mean complement sizes of Version 3 are higher than in Version 1. He evaluated, however, only the mean values, but looking closely at the results suggests that the median values could be in favour of Version 3. Therefore, we decided to reinvestigate this question. Indeed, in our own results the median complement sizes of Version 3 are considerably lower than the ones of Version 1 and Version 2. We will further elaborate on this point in Chapter~\ref{chap_results}.

Versions 4 and 5 in our above listing are for investigating the M1 and M2 optimisations. As M2 requires M1, there are only these two possible combinations. Versions 6 and 7 then enhance the ``better'' one of Version 4 and 5 with R2C and its alternative R2C+C. As we will see in Chapter~\ref{chap_results}, the better one of Version 4 and 5 in terms of median complement sizes is Version 4. That is, the application of M2 results in a decline, rather than a gain, in performance compared to the application of M1 alone. We have to note at this point that such results are always specific to the used the test set, and not universally valid. With a different test set, Version 5 might indeed be better than Version 4. As we will see in the next section, this is the case for our alternative test set consisting of the first four Michel automata.

Version 8, finally, is again the plain Fribourg construction, but this time the output automata are reduced by removing their unreachable and dead states. Comparing the results of Version 8 with Version 1 gives an idea of how many unreachable and dead states the Fribourg construction produces. This is inspired by the paper of the GOAL authors~\cite{2011_tsai} in which the number of unreachable and dead states is one of the main metrics for assessing the performance of a construction.

% As in Section~\ref{optimisations}, we refer to the first optimisation as R2C, the second one as M1, and the third one as M2. These optimisations have the following dependencies:
% \begin{enumerate}
% \item R2C can only be applied if the input automaton is complete
% \item M2 can only be applied if M1 is also applied
% \end{enumerate}

% Regarding the dependency of R2C, there are two possibilities. First (\em{R2C-A}), the R2C optimisation is selectively applied to the input automata which are complete, and not to the others. Second (\em{R2C-B}), all automata are made complete beforehand (by adding a sink state), and then the R2C optimisation is applied to all the automata.

% Göttel ~\cite{2013_bsc_goettel} has compared \em{R2C-B} to a plain version of the Fribourg construction where no optimisations at all are applied. He used the same test data as we do. The result was that \em{R2C-B} produces on average slightly less states, but the peak number of generated states are higher than in the plain version. It will be interesting to see if we can replicate these results, and how the selective application of the R2C optimisation (\em{R2C-A}) performs compared to \em{R2C-B}.

% Regarding the dependencies of the M1 and M2 optimisation, there are only two cases we can test. First, M1 alone, and second, M1 and M2 together. Assuming that the R2C optimisation adds a certain performance gain on top of an existing construction, we can then combine the better one of \em{M1} and \em{M1+M2} with R2C. We can already reveal at this point that \em{M1} performs slightly better than \em{M1+M2} on our test set, even though \em{M1+M2} has a better theoretical worst-case complexity. This topic is further discussed in Chapter~\ref{chap_results}. Thus, the version that we will want to test is \em{M1+R2C}.

% Furthermore, we can also investigate the effect of some generic optimisations on our construction. The most generic optimisations, which are included in most complementation constructions in GOAL, and also our plugin with the Fribourg construction, are:
% \begin{enumerate}
% \item Maximise the acceptance set of the input automaton (MACC)
% \item Remove unreachable and dead states from the output automaton (R)
% \end{enumerate}

% By applying these two optimisations to the best version of the Fribourg construction, we can see how far we can go with tweaking our construction, with respect to our set of test automata.

% Summarising, for the internal tests we are going to carry out runs of the following versions of the Fribourg construction:
% \begin{enumerate}
% \item \em{Fribourg}
% \item \em{Fribourg+R2C}
% \item \em{Fribourg+R2C+C}
% \item \em{Fribourg+M1}
% \item \em{Fribourg+M1+M2}
% \item \em{Fribourg+M1+R2C}
% \item \em{Fribourg+M1+R2C+MACC+R}
% \end{enumerate}

\subsubsection{Michel Test Set}
The versions we tested for the Michel test set are the following:
\begin{enumerate}
\item Fribourg
\item Fribourg+R2C
\item Fribourg+M1
\item Fribourg+M1+M2
\item Fribourg+M1+M2+R2C
\item Fribourg+R
\end{enumerate}

The rationale for choosing these versions is basically the same as for the GOAL test set with the following differences. First, Michel automata are complete, thus it is not necessary to apply the C option. Rather, R2C will automatically apply to all automata. Second, the aim of Version 5 is again to enhance the better one of Versions 3 and 4 with R2C. However, contrarily to the \goal{} test set Fribourg+M1+M2 is better than Fribourg+M1 for the Michel automata. That is why Version 5 is Fribourg+M1+M2+R2C.


\subsection{External Tests}
In the so called external tests we compare the best version of the Fribourg construction with different complementation constructions, again for both the GOAL test set and the Michel test set. The concrete constructions we compared for the external tests are the following.

\begin{enumerate}
\item Piterman+EQ+RO
\item Slice+P+RO+MADJ+EG
\item Rank+TR+RO
\item Fribourg+M1+R2C (\goal{} test set) and Fribourg+M1+M2+R2C (Michel test set)
\end{enumerate}

For the alternative constructions, we chose the Piterman, Slice, and Rank construction. These constructions are representative for three of the four main complementation approaches, determinization-based, slice-based, and rank-based. The fourth complementation approach is Ramsey-based and there is an implementation of the Ramsey construction in \goal{}. However, in preliminary tests, we realised that this construction is not performant enough to be used on our test sets within our time and memory constraints. The authors of \goal{} came to a similar conclusion when they made similar experiments on the \goal{} test set~\cite{2011_tsai}. In their case, the Ramsey construction could not complete any of the 11,000 automata within the set time and memory constraints, and they went on to do the result analysis without the Ramysey construction.

The same applies to all the other constructions that are implemented in \goal{}. We did preliminary tests with all of them and saw that only the three mentioned constructions, Piterman, Slice, and Rank, can be resonably used on the \goal{} test set\footnote{The Safra construction would also have been possible, but the Safra construction is similar to the Piterman construction.}.

The three chosen constructions also have optimisations in their implementations in \goal{}. To have a fair comparison to the best version of the Fribourg construction, which also uses optimisations, we activated the optimisation of these constructions as well. In particular, we chose those optimisations which are set as default in the \goal{} GUI for each construction.

We use Fribourg+M1+R2C for the \goal{} test set and Fribourg+M1+M2+R2C for the Michel test set. This is because these two versions are the most performant ones for the respective test sets.


% \textbf{Justification why to use only piterman, slice, and rank}$\\$
% Notes in UBELIX/jobs/2014-10-09: $\\$
% Made test with complementing the first 10 of the size 15 test set with all constructions, and only piterman, slice, and rank (and safra) completed all of them. $\\$
% See Tsai (2011)~\cite{2011_tsai} page 5: they compared ramsey, piterman, rank, and slice. But ramsey couldn't complement any of the 11,000 automata of size 15. $\\$
% Ramsey, piterman, rank, and slice are representative for the four main complementatio approaches: Ramsey-based, determinization-based, rank-based, and slice-based.


\subsection{Time and Memory Limits}
We defined a time limit of 600 seconds CPU time and a memory limit of 1 GB per complementation task in the GOAL test set. That means, if the complementation of a single automaton is not finished after 600 seconds CPU time or uses more than 1 GB memory, the task is aborted and marked as a \textit{timeout} or \textit{memory excess}.

These limits correspond to the ones used by the experiments of the \goal{} authors~\cite{2011_tsai}. However, from their paper it is not clear if their time limit is in CPU time or wallclock time. But since they used different computing nodes, our results regarding the number of timeouts will anyway differ from theirs.

The reason for these limits is simply the restricted amount of time and memory resources that we have available for the experiments. In an ideal world, we would let every complementation task run to its end, no matter how long it takes and how much memory it uses. This would give a perfectly unbiased picture of the results. By setting time and memory limits, we basically exclude the most difficult automata from the experiment. However, as mentioned, the practical reasons of limited time and computing power force as to make this compromise.

The timeout and memory limit of 1 GB apply just to the automata of the \goal{} test set. For the Michel test set we did not set a timeout because we wanted each one of the four automata to finish. The longest complementation task of a Michel automaton consequently durated 109,810 seconds which is about 28 hours. We set a very high memory limit of 14 GB for the Michel test set to avoid memory excesses as we wanted each automaton to successfully complete. The number of 14 GB is determined by the physically available memory on the used computing nodes. All Michel automata successfully completed with this amount of memory.

The timeout was implemented by the means of the \textsf{ulimit} Bash builtin\footnote{\url{http://linux.die.net/man/1/bash}}, which allows to set a maximum time after which running processes are killed. The memory limit was implemented by setting the maximum size of the Java heap, which can be done by the \textsf{-Xmx} option to the Java Virtual Machine (JVM). The heap is the main memory area of Java and the place where all the objects reside. Note that since our memory limit defines actually the size of the Java heap, the total amount of memory used by the process is higher than our limit, as Java has some other memory areas, for example for the JVM itself. However, this is a rather constant amount of memory and independent from the current automaton, so it does not disturb the relative comparisons of the results.

The presence of aborted complementation task require the consideration of the so called effective samples in the result analysis, as introduced in the experiment paper of the \goal{} authors. The effective samples are those automata which have been successfully completed by \textit{all} constructions that are to be compared to each other. Imagine two constructions $A$ and $B$ where $A$ is successful complementing all the automata, whereas $B$ has timeouts or memory excesses at 100 of the automata. If we would now take, for example, the median complement sizes of the two result sets without first extracting the effective samples, then $B$ is likely be assessed as too good relative to $A$, because $B$'s results do not include the 100 automata at which it failed, and which are thus likely to have large complement sizes with $B$. The same 100 automata would however be included in the results of $A$. Therefore, all the result analysis of the experiments with the \goal{} test sets, that we present in Chapter~\ref{chap_results}, are based on the effective samples of the result sets. 


\subsection{Execution Environment}
We executed the experiments on a high performance computing (HPC) computer cluster called UBELIX at the University of Bern\footnote{\url{http://ubelix.unibe.ch}}. This cluster consists of different types of Linux-driven HPC computing nodes, and is managed by Oracle Grid Engine\footnote{\url{http://www.oracle.com/us/products/tools/oracle-grid-engine-075549.html}} (formerly known as Sun Grid Engine) version 6.2. Oracle Grid Engine is a so called load scheduler that is responsible for automatically distributing computing tasks to computing nodes.

The basic workflow of working on the cluster is to prepare a so called job and specify the resources that the job requires for running (time, memory, number of CPU cores, and so on). Then the job can be submitted to the grid engine. The grid engine first puts incoming jobs in a queue and then automatically dispatches them to suitable computing nodes as soon as the required capacity is available.

A job in our case is the execution of a construction (or version of the Fribourg construction) over an entire test set. Thus, we were running 22 jobs in total. We arranged that all jobs were run on identical nodes with the following specifications:

\begin{itemize}
\item Processor: Intel Xeon E5-2665 2.40GHz (64 bit)
\item CPU cores: 16
\item Operating System: Red Hat Enterprise Linux 6.6
\end{itemize}

Since we use the execution time (CPU time) as a secondary metrics, next to the complement sizes, it is important that all complementation tasks are executed on the same type of hardware. \goal{} is multithreaded and thus the jobs may use multiple CPU cores. The number of CPU cores a job may use is not restricted (up to the number of available cores on the node, in our case 16), but our observation is that the jobs use a rather small number of cores (2--3). Note that the measurement of the execution time is not affected by the number of cores a process uses, as the CPU times are measured separately on each core and then added together.  

% \begin{itemize}
% \item mpi.q
%   \begin{itemize}
%   \item hnode 01--42
%   \item Intel Xeon E5-2665 2.40GHz
%   \item 16 CPU cores (slots)
%   \item 64 GB RAM
%   \item $\rightarrow$ 4 GB RAM per core (slot)
%   \item h\_cpu limit: 72:00:00
%   \item h\_rt limit: 73:00:00
%   \end{itemize}
% \item highmem.q
%   \begin{itemize}
%   \item jnode 01--21
%   \item Intel Xeon E5-2665 2.40GHz
%   \item 16 CPU cores (slots)
%   \item 256 GB RAM
%   \item $\rightarrow$ 16 GB RAM per core (slot)
%   \end{itemize}
% \end{itemize}


% \begin{enumerate}
% \item Internal on GOAL testset
% \item External on GOAL testset
% \item Internal on Michel automata
% \item External on Michel automata
% \item Completeness of GOAL testset
% \item Universality of GOAL testset
% \end{enumerate}

% The tests are successful with the following resources:

% \begin{tabular}{|p{1.5cm}|l|r|r|r|r|r|p{3.5cm}|}
% \hline
% Test & Queue & Slots & \parbox[t]{1.75cm}{Job\\memory\\limit} & \parbox[t]{1.75cm}{Job CPU\\time limit} & \parbox[t]{1.75cm}{CPU time\\limit per\\automaton\\} & \parbox[t]{1.75cm}{Memory\\limit per\\automaton\\} & Notes \\
% \hline
% 1 and 2 & mpi.q & 4 & 4 GB & 72:00:00 & 600 sec. & 1 GB & rank -tr -ro has to be run on 10 partitions of the test set \\
% \hline
% 3 and 4 & highmem.q & 4 & 16 GB & 72:00:00 & None & 14 GB & piterman -eq -sim -ro out of memory on Michel N4 \\
% \hline
% 4 & mpi.q & 4 & 4 GB & 72:00:00 & None & 1 GB & \\
% \hline
% 5 & mpi.q & 4 & 4 GB & 72:00:00 & None & 2 GB & universal -m piterman -eq -ro \\
% \hline
% \end{tabular}
