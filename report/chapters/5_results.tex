\section{Internal Tests}

Note: all results from UBELIX:~/jobs/2015-05-16

\subsection{GOAL Test Set}

\subsubsection{Timeouts and Memory Excesses}

Timeout for each complementation task: 600 seconds CPU time

Memory limit: Java heap set to 1 GB

\begin{tabular}{|l|rr|}
\hline
Version & Timeouts & Memory Excesses \\
\hline
\em{Fribourg}               & 48 & 0 \\
\em{Fribourg+R2C}           & 30 & 0 \\
\em{Fribourg+R2C+C}         & 54 & 0 \\  
\em{Fribourg+M1}            &  2 & 0 \\
\em{Fribourg+M1+M2}         &  1 & 0 \\
\em{Fribourg+M1+R2C}        &  1 & 0 \\
\em{Fribourg+M1+R2C+MACC+R} &  1 & 0 \\
\hline
\end{tabular}

Effective samples: 10,939

\subsubsection{Statistics Aggregated}
\begin{table}[ht]
\centering
\begin{tabular}{lrrrrrr}
  \hline
Version & Mean & Min & P25 & Median & P75 & Max \\ 
  \hline
Fribourg & 2,004.6 & 2 & 222.0 & 761.0 & 2,175.0 & 37,904 \\ 
  Fribourg+R2C & 1,955.9 & 2 & 180.0 & 689.0 & 2,127.5 & 37,904 \\ 
  Fribourg+R2C+C & 2,424.6 & 2 & 85.0 & 451.0 & 2,329.0 & 54,648 \\ 
  Fribourg+M1 & 963.2 & 2 & 177.0 & 482.0 & 1,138.0 & 16,260 \\ 
  Fribourg+M1+M2 & 958.0 & 2 & 181.0 & 496.0 & 1,156.5 & 15,223 \\ 
  Fribourg+M1+R2C & 937.7 & 2 & 152.0 & 447.0 & 1,118.0 & 16,260 \\ 
  Fribourg+M1+R2C+MACC+R & 115.4 & 1 & 1.0 & 1.0 & 20.0 & 9,843 \\ 
   \hline
\end{tabular}
\end{table}

\begin{table}[ht]
\centering
\begin{tabular}{lrrrrrr}
  \toprule
Version & Mean & Min & P25 & Median & P75 & Max \\ 
  \midrule
Fribourg & 2,004.6 & 2 & 222.0 & 761.0 & 2,175.0 & 37,904 \\ 
  Fribourg+R2C & 1,955.9 & 2 & 180.0 & 689.0 & 2,127.5 & 37,904 \\ 
  Fribourg+R2C+C & 2,424.6 & 2 & 85.0 & 451.0 & 2,329.0 & 54,648 \\ 
  Fribourg+M1 & 963.2 & 2 & 177.0 & 482.0 & 1,138.0 & 16,260 \\ 
  Fribourg+M1+M2 & 958.0 & 2 & 181.0 & 496.0 & 1,156.5 & 15,223 \\ 
  Fribourg+M1+R2C & 937.7 & 2 & 152.0 & 447.0 & 1,118.0 & 16,260 \\ 
  Fribourg+M1+R2C+MACC+R & 115.4 & 1 & 1.0 & 1.0 & 20.0 & 9,843 \\ 
   \bottomrule
\end{tabular}
\end{table}

Idea: R2C+C has a muc lower median than R2C. So, it would be interesting to see how M1+R2C+C compares to M1+R2C. Furthermore, M1+R2C+MACC+R is dominated by R. So, we could apply R to the best of M1+R2C and M1+R2C+C. I.e.
\begin{itemize}
\item Add Fribourg+M1+R2C+C
\item Drop Fribourg+M1+R2C+MACC+R
\item Add Fribourg+M1+R2C+R or Fribourg+M1+R2C+C+R
\end{itemize}



\subsection{Michel Automata}
Michel automata are complete, thus the C options makes no sense.

For the Michel automata Fribourg+M1+M2 is better than Fribourg+M1. Fribourg+M1+R2C is again much better than Fribourg+M1 and even Fribourg+M1+M2. Thus, we have to test Fribourg+M1+M2+R2C. 

\begin{itemize}
\item Add Fribourg+M1+M2+R2C
\item Add Fribourg+M1+M2+R2C+R
\item Add Fribourg+R
\end{itemize}



\begin{tabular}{r|rrrrrrrrrr}
  \hline
 & 0.1 & 0.2 & 0.3 & 0.4 & 0.5 & 0.6 & 0.7 & 0.8 & 0.9 & 1.0 \\ 
  \hline
1.0 & 580.3 & 465.0 & 786.4 & 401.2 & 398.4 & 300.3 & 336.7 & 263.3 & 297.4 & 77.3 \\ 
  1.2 & 1730.8 & 2050.2 & 1780.7 & 1808.5 & 1328.1 & 1298.1 & 1421.8 & 1108.0 & 987.4 & 136.4 \\ 
  1.4 & 3409.1 & 3251.7 & 2471.2 & 3052.9 & 2210.9 & 2255.1 & 1638.0 & 1437.3 & 1561.8 & 172.0 \\ 
  1.6 & 3493.9 & 3280.5 & 3395.7 & 3005.0 & 2215.7 & 1921.1 & 1599.5 & 1418.9 & 1401.5 & 170.3 \\ 
  1.8 & 2714.1 & 2418.7 & 2677.5 & 2199.3 & 2102.1 & 1689.1 & 1367.8 & 876.0 & 842.8 & 127.3 \\ 
  2.0 & 1778.3 & 1961.6 & 1769.2 & 1805.8 & 1401.0 & 1275.3 & 803.5 & 682.4 & 510.7 & 107.8 \\ 
  2.2 & 1180.9 & 1498.6 & 1257.8 & 1219.9 & 989.9 & 926.0 & 519.3 & 412.0 & 314.4 & 82.6 \\ 
  2.4 & 719.6 & 992.4 & 1036.3 & 848.8 & 659.2 & 615.9 & 388.6 & 287.8 & 195.4 & 60.8 \\ 
  2.6 & 553.0 & 682.8 & 673.3 & 660.3 & 555.1 & 483.2 & 265.0 & 190.2 & 153.2 & 49.1 \\ 
  2.8 & 454.6 & 490.5 & 609.6 & 489.8 & 419.2 & 336.3 & 222.7 & 154.1 & 118.9 & 41.1 \\ 
  3.0 & 276.4 & 381.4 & 403.5 & 390.1 & 323.8 & 255.3 & 136.2 & 119.5 & 82.2 & 35.0 \\ 
   \hline
\end{tabular}

asdflkjasdf


asdflkjf

\begin{tabular}{r|rrrrrrrrrr}
 & 0.1 & 0.2 & 0.3 & 0.4 & 0.5 & 0.6 & 0.7 & 0.8 & 0.9 & 1.0 \\ 
  \hline
1.0 & 269.0 & 308.0 & 254.0 & 236.0 & 238.5 & 297.0 & 266.0 & 156.0 & 207.0 & 68.0 \\ 
  1.2 & 960.0 & 1,407.5 & 1,479.0 & 2,150.0 & 1,152.0 & 1,090.5 & 942.5 & 1,206.5 & 718.0 & 104.5 \\ 
  1.4 & 3,426.0 & 2,915.0 & 2,752.0 & 3,393.0 & 2,693.0 & 3,265.5 & 2,263.5 & 2,425.0 & 1,844.5 & 154.5 \\ 
  1.6 & 3,799.0 & 3,698.0 & 4,901.5 & 3,926.0 & 3,960.0 & 3,655.0 & 2,580.5 & 1,905.5 & 2,124.5 & 155.0 \\ 
  1.8 & 3,375.0 & 3,169.0 & 3,420.5 & 3,967.0 & 3,943.0 & 3,132.0 & 2,246.5 & 1,144.0 & 971.5 & 114.0 \\ 
  2.0 & 1,906.5 & 2,261.0 & 2,383.0 & 2,884.0 & 2,354.5 & 2,096.5 & 1,169.5 & 932.0 & 568.0 & 98.5 \\ 
  2.2 & 1,467.0 & 1,633.0 & 1,795.5 & 1,942.5 & 1,611.5 & 1,640.5 & 569.5 & 499.0 & 330.5 & 78.5 \\ 
  2.4 & 924.5 & 1,232.5 & 1,319.0 & 1,317.5 & 1,056.5 & 886.5 & 514.5 & 314.5 & 182.0 & 59.0 \\ 
  2.6 & 625.0 & 763.0 & 880.5 & 945.5 & 828.0 & 684.5 & 316.0 & 175.0 & 132.0 & 44.5 \\ 
  2.8 & 483.5 & 584.5 & 836.0 & 690.0 & 575.0 & 395.5 & 240.0 & 151.5 & 103.0 & 41.0 \\ 
  3.0 & 319.5 & 450.5 & 557.0 & 523.5 & 367.5 & 313.5 & 155.5 & 116.0 & 84.5 & 32.0 \\ 
\end{tabular}

\begin{tabular}{lrrrrrr}
  \hline
Version & Mean & Min & P25 & Median & P75 & Max \\ 
  \hline
fribourg.goal & 2,004.57 & 2 & 222.00 & 761.00 & 2,175.00 & 37,904 \\ 
  fribourg.m1.goal & 963.17 & 2 & 177.00 & 482.00 & 1,138.00 & 16,260 \\ 
  fribourg.m1.m2.goal & 958.00 & 2 & 181.00 & 496.00 & 1,156.50 & 15,223 \\ 
  fribourg.m1.r2c.goal & 937.66 & 2 & 152.00 & 447.00 & 1,118.00 & 16,260 \\ 
  fribourg.m1.r2c.macc.r.goal & 115.40 & 1 & 1.00 & 1.00 & 20.00 & 9,843 \\ 
  fribourg.r2c.c.goal & 2,424.56 & 2 & 85.00 & 451.00 & 2,329.00 & 54,648 \\ 
  fribourg.r2c.goal & 1,955.91 & 2 & 180.00 & 689.00 & 2,127.50 & 37,904 \\ 
   \hline
\end{tabular}

% \begin{figure}
%     \centering
%     \begin{subfigure}[t]{0.45\textwidth}
%         \centering
%         \includegraphics[width=\textwidth]{/Users/dw/Desktop/plot1.pdf}
%         \caption{Because of rank, we would exclude the most difficult cases for the other constructions from the analysis.}
%         \label{fig:y equals x}
%     \end{subfigure}
%     \hfill
%     \begin{subfigure}[t]{0.45\textwidth}
%         \centering
%         \includegraphics[width=\textwidth]{/Users/dw/Desktop/plot2.pdf}
%         \caption{$y=3sinx$}
%         \label{fig:three sin x}
%     \end{subfigure}

%     \begin{subfigure}[t]{0.45\textwidth}
%         \centering
%         \includegraphics[width=\textwidth]{/Users/dw/Desktop/plot3.pdf}
%         \caption{$y=5/x$}
%         \label{fig:five over x}
%     \end{subfigure}
%     \hfill
%     \begin{subfigure}[t]{0.45\textwidth}
%         \centering
%         \includegraphics[width=\textwidth]{/Users/dw/Desktop/plot3.pdf}
%         \caption{$y=5/x$}
%         \label{fig:five over x}
%     \end{subfigure}
%     \caption{Three simple graphs}
%     \label{fig:three graphs}
% \end{figure}



\section{External Tests}

\subsection{GOAL Test Set}

With \em{rank -tr -ro} there are 3796 uncompleted tasks (7,204 effective samples), without it only 7 (10,993 effective samples). Should the main analysis be done without rank? Because of rank, we would exclude the most difficult cases for the other constructions from the analysis.

With \em{piterman -eq -sim -ro}, the median is 1 and the 75th percentile 21. It' similar as when removing unreachable and dead states from the output automaton. I think this is caused by the sim optimisation, which simplifies one of the intermediate automata of the construction. Have to try the following:

\begin{itemize}
\item piterman -eq -sim -ro -r
\item piterman -eq -ro
\end{itemize}
