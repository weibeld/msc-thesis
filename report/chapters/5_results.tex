\section{Internal Tests}

Note: all results from UBELIX:~/jobs/2015-05-16

\subsection{GOAL Test Set}

\subsubsection{Timeouts and Memory Excesses}

Timeout for each complementation task: 600 seconds CPU time

Memory limit: Java heap set to 1 GB

\begin{tabular}{|l|rr|}
\hline
Version & Timeouts & Memory Excesses \\
\hline
\em{Fribourg}               & 48 & 0 \\
\em{Fribourg+R2C}           & 30 & 0 \\
\em{Fribourg+R2C+C}         & 54 & 0 \\  
\em{Fribourg+M1}            &  2 & 0 \\
\em{Fribourg+M1+M2}         &  1 & 0 \\
\em{Fribourg+M1+R2C}        &  1 & 0 \\
\em{Fribourg+M1+R2C+MACC+R} &  1 & 0 \\
\hline
\end{tabular}

Effective samples: 10,939

\subsubsection{Statistics Aggregated}
\begin{table}[ht]
\centering
\begin{tabular}{lrrrrrr}
  \hline
Version & Mean & Min & P25 & Median & P75 & Max \\ 
  \hline
Fribourg & 2,004.6 & 2 & 222.0 & 761.0 & 2,175.0 & 37,904 \\ 
  Fribourg+R2C & 1,955.9 & 2 & 180.0 & 689.0 & 2,127.5 & 37,904 \\ 
  Fribourg+R2C+C & 2,424.6 & 2 & 85.0 & 451.0 & 2,329.0 & 54,648 \\ 
  Fribourg+M1 & 963.2 & 2 & 177.0 & 482.0 & 1,138.0 & 16,260 \\ 
  Fribourg+M1+M2 & 958.0 & 2 & 181.0 & 496.0 & 1,156.5 & 15,223 \\ 
  Fribourg+M1+R2C & 937.7 & 2 & 152.0 & 447.0 & 1,118.0 & 16,260 \\ 
  Fribourg+M1+R2C+MACC+R & 115.4 & 1 & 1.0 & 1.0 & 20.0 & 9,843 \\ 
   \hline
\end{tabular}
\end{table}

\begin{table}[ht]
\centering
\begin{tabular}{lrrrrrr}
  \toprule
Version & Mean & Min & P25 & Median & P75 & Max \\ 
  \midrule
Fribourg & 2,004.6 & 2 & 222.0 & 761.0 & 2,175.0 & 37,904 \\ 
  Fribourg+R2C & 1,955.9 & 2 & 180.0 & 689.0 & 2,127.5 & 37,904 \\ 
  Fribourg+R2C+C & 2,424.6 & 2 & 85.0 & 451.0 & 2,329.0 & 54,648 \\ 
  Fribourg+M1 & 963.2 & 2 & 177.0 & 482.0 & 1,138.0 & 16,260 \\ 
  Fribourg+M1+M2 & 958.0 & 2 & 181.0 & 496.0 & 1,156.5 & 15,223 \\ 
  Fribourg+M1+R2C & 937.7 & 2 & 152.0 & 447.0 & 1,118.0 & 16,260 \\ 
  Fribourg+M1+R2C+MACC+R & 115.4 & 1 & 1.0 & 1.0 & 20.0 & 9,843 \\ 
   \bottomrule
\end{tabular}
\end{table}

Idea: R2C+C has a muc lower median than R2C. So, it would be interesting to see how M1+R2C+C compares to M1+R2C. Furthermore, M1+R2C+MACC+R is dominated by R. So, we could apply R to the best of M1+R2C and M1+R2C+C. I.e.
\begin{itemize}
\item Add Fribourg+M1+R2C+C
\item Drop Fribourg+M1+R2C+MACC+R
\item Add Fribourg+M1+R2C+R or Fribourg+M1+R2C+C+R
\end{itemize}



\subsection{Michel Automata}
Are Michel automata complete?

C shouldn't be applied at all, because then the input automaton would not be a Michel automaton.



\begin{tabular}{r|rrrrrrrrrr}
  \hline
 & 0.1 & 0.2 & 0.3 & 0.4 & 0.5 & 0.6 & 0.7 & 0.8 & 0.9 & 1.0 \\ 
  \hline
1.0 & 580.3 & 465.0 & 786.4 & 401.2 & 398.4 & 300.3 & 336.7 & 263.3 & 297.4 & 77.3 \\ 
  1.2 & 1730.8 & 2050.2 & 1780.7 & 1808.5 & 1328.1 & 1298.1 & 1421.8 & 1108.0 & 987.4 & 136.4 \\ 
  1.4 & 3409.1 & 3251.7 & 2471.2 & 3052.9 & 2210.9 & 2255.1 & 1638.0 & 1437.3 & 1561.8 & 172.0 \\ 
  1.6 & 3493.9 & 3280.5 & 3395.7 & 3005.0 & 2215.7 & 1921.1 & 1599.5 & 1418.9 & 1401.5 & 170.3 \\ 
  1.8 & 2714.1 & 2418.7 & 2677.5 & 2199.3 & 2102.1 & 1689.1 & 1367.8 & 876.0 & 842.8 & 127.3 \\ 
  2.0 & 1778.3 & 1961.6 & 1769.2 & 1805.8 & 1401.0 & 1275.3 & 803.5 & 682.4 & 510.7 & 107.8 \\ 
  2.2 & 1180.9 & 1498.6 & 1257.8 & 1219.9 & 989.9 & 926.0 & 519.3 & 412.0 & 314.4 & 82.6 \\ 
  2.4 & 719.6 & 992.4 & 1036.3 & 848.8 & 659.2 & 615.9 & 388.6 & 287.8 & 195.4 & 60.8 \\ 
  2.6 & 553.0 & 682.8 & 673.3 & 660.3 & 555.1 & 483.2 & 265.0 & 190.2 & 153.2 & 49.1 \\ 
  2.8 & 454.6 & 490.5 & 609.6 & 489.8 & 419.2 & 336.3 & 222.7 & 154.1 & 118.9 & 41.1 \\ 
  3.0 & 276.4 & 381.4 & 403.5 & 390.1 & 323.8 & 255.3 & 136.2 & 119.5 & 82.2 & 35.0 \\ 
   \hline
\end{tabular}



\section{External Tests}
