\lettrine{T}{he aim of} this thesis was to empirically investigate the performance of the Fribourg construction, a slice-based Büchi complementation construction, in order to contribute to a better understanding of Büchi complementation. We implemented the Fribourg construction in the form of a plugin for the \goal{} tool. \goal{} is a tool for creating and manipulating different types of \om-automata, and it contains implementations of various Büchi complementation constructions. The installation of our plugin, integrates the Fribourg construction in \goal, and makes it accessible side-by-side to the other complementation constructions of \goal. 

The embedment of the Fribourg construction in \goal{} allows not only to investigate the performance Fribourg construction itself, but also to compare it to other complementation constructions. The comparisons of different variants of the Fribourg construction itself constituted the so-called internal tests. On the other hand, the comparison of the Fribourg construction to other complementation constructions constituted the so-called external tests. The constructions against which we compared the Fribourg construction are a determinisation-based (Piterman), rank-based (Rank), and another slice-based (Slice) construction.

Both, the internal and external tests, were carried out by testing the different constructions on a fixed set of input automata. We used two distinct test sets. First, the \goal{} test set, consisting of 11,000 automata of size 15 and with an alphabet size of 2. These automata are further divided into 110 classes consisting of combinations of 11 transition densities and 10 acceptance densities. Second, the Michel test set, which consists of four automata with 3 to 6 states, that have been used by Michel in 1988 to prove a lower bound of $n!$ for the worst-case state complexity of Büchi complementation. All these types of automata have been used as the test data in previous empirical studies.

The results of the investigation showed that the optimisations that have been proposed for the Fribourg construction clearly improve the performance of the construction. The application of these optimisations cuts the average complement size for the automata of the \goal{} test set in half. For the Michel automata, the application of the optimisations cause the complements to be multiple times (up to 6.5 times) smaller than without the optimisations.

Regarding the comparison of the Fribourg construction to other construction, our results show that the Fribourg construction (with its optimisations) can perfectly compete with them. For the \goal{} test set, the Fribourg construction is ranked third, with a large margin to the first-placed Piterman, a small margin to the second-placed Slice, and a large margin to the fourth-placed Rank. For the Michel test set, the Fribourg construction can be ranked second, with a moderated margin to the first-placed Rank, a considerable margin to the third-placed Slice, and a large margin to the fourth-placed Piterman.

Besides these results about the performance of the Fribourg construction, our study provided several insights into Büchi complementation in general. First, the results of our tests on the \goal{} test set give an idea of expectable complement sizes that the tested constructions produce on automata with 15 states and an alphabet size of 2. With a probability of 50\% such automata result typically in a complement with some hundred states. These expectations can be further refined with our results by taking into account the transition density and acceptance density of the automata.

Second, our study allowed to identify characteristics, based on transition density and acceptance density, that make automata by trend harder or easier to complement. For the \goal{} test set, automata with a transition density between 1.4 and 1.8, and an acceptance density up to 0.6 are the hardest automata. The easiest automata are those with either a very high or low transition density, or a high acceptance density. Generally, a higher acceptance density simplifies the complementation task. These results apply specifically to the automata of the \goal{} test set, and it is subject to further research if and how they can be generalised to other automata.

Next, our results confirm that the worst-case state complexity of a construction is not instructive regarding the actual performance of a construction. We observed that in many cases constructions with a lower worst-case state complexity perform worse than other constructions on large sets of automata. There seems to be no relation between the worst-case performance of a construction and its actual performance.

Furthermore, our results also confirm that there is most likely no overall-best complementation construction. That is, a complementation construction that is better for \textit{every} complementation task than other constructions. Rather each construction has its strengths and weaknesses and might be better than other constructions on one type of automata, but worse on another type of automata. The investigation of these relative strengths of complementation constructions is a promising direction of future research.

We identified the limitations of our study and how they affect the drawing of conclusions. These limitations include the limited test data, the reliance on implementations that empirical studies generally exhibit, the test setup, and the depth of the statistical evaluation of the results.

Regarding future work, we think that a good idea is to start with tackling the limitations of the present study. Such an improvd study could concretise the insights that we gained or confirmed as a result of this thesis. In addition, we think that the data that we collected has the potential to serve as the basis for further studies. As mentioned, our statistical evaluation of this data was not exhaustive, and a more in-depth, but narrower-focused analysis, on the same data could reveal interesting new insights.

