In the following two figures, we provide the matrices corresponding to all the perspective plots in Chapter~\ref{chap_results} (that is, the following matrices contain the same data as the corrsponding perspective plots, but just visualised in a different way).

The matrices in Figure~\ref{e.g.matrices} correspond to the perspective plots in Figure~\ref{e.g.persp} (a) and (b) (external tests on the \goal{} test set). The eight matrices in Figure~\ref{i.g.matrices} correspond to the perspective plots of Figure~\ref{i.g.persp_1} and~\ref{i.g.persp_2} (internal tests on the \goal{} test set).

The cells of the matrices represent the median complement sizes of all the transition density/acceptance density classes of the \goal{} test set. The rows of a matrix represent the transition densities, and the columns represent the acceptance densities.

Note that in Figure~\ref{e.g.matrices}, we do not provide the matrix for Fribourg+M1+R2C, because this is the same matrix as the one in Figure~\ref{i.g.matrices} (f).


% Fixed width right-aligned table column
\newcolumntype{R}{>{\raggedleft\arraybackslash}p{2.25em}}

% Table spacings
\renewcommand{\arraystretch}{1.4}

% Width of a subtable (fraction of \textwidth)
\renewcommand{\subwidth}{0.48}

% Force single floating figure to top of page
% \makeatletter
% \setlength{\@fptop}{0pt}
% \makeatother

\vskip1.5cm

\begin{figure}[htb]
  \centering
  \begin{scriptsize}
  \renewcommand{\tabcolsep}{0.05cm}
  \begin{subtable}[t]{\subwidth\textwidth}
    \centering
    % latex table generated in R 3.1.2 by xtable 1.7-4 package
% Sun Aug 16 15:57:21 2015
\begin{tabular}{r|RRRRRRRRRR}
  & 0.1 & 0.2 & 0.3 & 0.4 & 0.5 & 0.6 & 0.7 & 0.8 & 0.9 & 1.0 \\ 
  \hline
1.0 & 130 & 117 & 109 & 77 & 69 & 61 & 56 & 40 & 40 & 29 \\ 
  1.2 & 387 & 456 & 352 & 281 & 155 & 136 & 101 & 105 & 75 & 45 \\ 
  1.4 & 822 & 683 & 394 & 376 & 230 & 204 & 151 & 120 & 105 & 63 \\ 
  1.6 & 890 & 594 & 458 & 321 & 237 & 178 & 134 & 114 & 113 & 61 \\ 
  1.8 & 624 & 507 & 324 & 275 & 196 & 136 & 110 & 92 & 89 & 41 \\ 
  2.0 & 362 & 286 & 211 & 176 & 117 & 103 & 79 & 64 & 59 & 34 \\ 
  2.2 & 248 & 222 & 124 & 116 & 82 & 73 & 56 & 52 & 50 & 28 \\ 
  2.4 & 147 & 145 & 114 & 87 & 56 & 48 & 43 & 39 & 35 & 19 \\ 
  2.6 & 115 & 117 & 67 & 61 & 47 & 42 & 32 & 29 & 29 & 15 \\ 
  2.8 & 95 & 71 & 52 & 45 & 38 & 29 & 27 & 25 & 23 & 13 \\ 
  3.0 & 59 & 60 & 47 & 35 & 32 & 27 & 22 & 21 & 20 & 10 \\ 
  \end{tabular}

    \caption{Piterman+EQ+RO}
  \end{subtable}
  \hfill
  \begin{subtable}[t]{\subwidth\textwidth}
    \centering
    % latex table generated in R 3.1.2 by xtable 1.7-4 package
% Sun Aug 16 15:57:21 2015
\begin{tabular}{r|RRRRRRRRRR}
  & 0.1 & 0.2 & 0.3 & 0.4 & 0.5 & 0.6 & 0.7 & 0.8 & 0.9 & 1.0 \\ 
  \hline
1.0 & 171 & 174 & 166 & 124 & 118 & 117 & 100 & 67 & 84 & 35 \\ 
  1.2 & 622 & 833 & 803 & 877 & 529 & 398 & 320 & 372 & 215 & 53 \\ 
  1.4 & 2,086 & 1,618 & 1,367 & 1,676 & 1,065 & 967 & 664 & 682 & 494 & 78 \\ 
  1.6 & 2,465 & 2,073 & 2,182 & 1,959 & 1,518 & 1,259 & 767 & 545 & 623 & 78 \\ 
  1.8 & 2,310 & 1,963 & 1,950 & 1,988 & 1,485 & 1,095 & 746 & 418 & 346 & 57 \\ 
  2.0 & 1,318 & 1,482 & 1,393 & 1,461 & 981 & 871 & 434 & 338 & 228 & 50 \\ 
  2.2 & 1,068 & 1,145 & 1,085 & 1,067 & 772 & 747 & 263 & 235 & 158 & 40 \\ 
  2.4 & 689 & 838 & 809 & 751 & 524 & 466 & 240 & 159 & 93 & 30 \\ 
  2.6 & 469 & 531 & 555 & 565 & 437 & 360 & 169 & 94 & 71 & 23 \\ 
  2.8 & 369 & 421 & 536 & 405 & 329 & 224 & 130 & 81 & 58 & 21 \\ 
  3.0 & 244 & 327 & 360 & 322 & 219 & 176 & 85 & 64 & 49 & 16 \\ 
  \end{tabular}

    \caption{Slice+P+RO+MADJ+EG}
  \end{subtable}
\end{scriptsize}
\caption{Median complement sizes based on the 10,998 effective samples of the external tests (without Rank) for each of the 110 transition density/acceptance density classes of the \goal{} test set.}
\label{e.g.matrices}
\end{figure}

\begin{figure}[ht]
  \centering
  \begin{scriptsize}
  \renewcommand{\tabcolsep}{0.05cm}
  \begin{subtable}[t]{\subwidth\textwidth}
    \centering
    % latex table generated in R 3.1.2 by xtable 1.7-4 package
% Wed Aug 19 09:24:32 2015
\begin{tabular}{r|RRRRRRRRRR}
  & 0.1 & 0.2 & 0.3 & 0.4 & 0.5 & 0.6 & 0.7 & 0.8 & 0.9 & 1.0 \\ 
  \hline
1.0 & 269 & 308 & 254 & 236 & 238 & 297 & 266 & 156 & 207 & 68 \\ 
  1.2 & 960 & 1,407 & 1,479 & 2,150 & 1,152 & 1,090 & 942 & 1,206 & 718 & 104 \\ 
  1.4 & 3,426 & 2,915 & 2,752 & 3,393 & 2,693 & 3,265 & 2,263 & 2,425 & 1,844 & 154 \\ 
  1.6 & 3,799 & 3,698 & 4,901 & 3,926 & 3,960 & 3,655 & 2,580 & 1,905 & 2,124 & 155 \\ 
  1.8 & 3,375 & 3,169 & 3,420 & 3,967 & 3,943 & 3,132 & 2,246 & 1,144 & 971 & 114 \\ 
  2.0 & 1,906 & 2,261 & 2,383 & 2,884 & 2,354 & 2,096 & 1,169 & 932 & 568 & 98 \\ 
  2.2 & 1,467 & 1,633 & 1,795 & 1,942 & 1,611 & 1,640 & 569 & 499 & 330 & 78 \\ 
  2.4 & 924 & 1,232 & 1,319 & 1,317 & 1,056 & 886 & 514 & 314 & 182 & 59 \\ 
  2.6 & 625 & 763 & 880 & 945 & 828 & 684 & 316 & 175 & 132 & 44 \\ 
  2.8 & 483 & 584 & 836 & 690 & 575 & 395 & 240 & 151 & 103 & 41 \\ 
  3.0 & 319 & 450 & 557 & 523 & 367 & 313 & 155 & 116 & 84 & 32 \\ 
  \end{tabular}

    \caption{Fribourg}
  \end{subtable}
  \hfill
  \begin{subtable}[t]{\subwidth\textwidth}
    \centering
    % latex table generated in R 3.1.2 by xtable 1.7-4 package
% Fri Jun 12 16:02:09 2015
\begin{tabular}{r|RRRRRRRRRR}
  & 0.1 & 0.2 & 0.3 & 0.4 & 0.5 & 0.6 & 0.7 & 0.8 & 0.9 & 1.0 \\ 
  \hline
1.0 & 269 & 308 & 254 & 236 & 238 & 297 & 266 & 156 & 207 & 68 \\ 
  1.2 & 960 & 1,407 & 1,479 & 2,150 & 1,152 & 1,090 & 942 & 1,206 & 718 & 104 \\ 
  1.4 & 3,426 & 2,915 & 2,752 & 3,393 & 2,693 & 3,265 & 2,263 & 2,425 & 1,844 & 154 \\ 
  1.6 & 3,799 & 3,698 & 4,901 & 3,926 & 3,960 & 3,655 & 2,580 & 1,905 & 2,124 & 155 \\ 
  1.8 & 3,375 & 3,169 & 3,420 & 3,967 & 3,943 & 3,093 & 2,246 & 1,144 & 971 & 114 \\ 
  2.0 & 1,906 & 2,184 & 2,383 & 2,818 & 2,354 & 1,989 & 1,127 & 885 & 568 & 97 \\ 
  2.2 & 1,410 & 1,561 & 1,639 & 1,884 & 1,609 & 1,588 & 496 & 464 & 284 & 78 \\ 
  2.4 & 884 & 1,200 & 1,234 & 1,184 & 939 & 806 & 373 & 256 & 165 & 55 \\ 
  2.6 & 575 & 731 & 815 & 860 & 751 & 575 & 246 & 162 & 114 & 43 \\ 
  2.8 & 431 & 530 & 672 & 466 & 371 & 274 & 174 & 120 & 85 & 36 \\ 
  3.0 & 232 & 325 & 344 & 360 & 269 & 169 & 91 & 85 & 53 & 27 \\ 
  \end{tabular}

    \caption{Fribourg+R2C}
  \end{subtable}

  \renewcommand{\tabcolsep}{0.05cm}
  \begin{subtable}[t]{\subwidth\textwidth}
    \centering
    % latex table generated in R 3.1.2 by xtable 1.7-4 package
% Fri Jun 12 16:02:09 2015
\begin{tabular}{r|RRRRRRRRRR}
  & 0.1 & 0.2 & 0.3 & 0.4 & 0.5 & 0.6 & 0.7 & 0.8 & 0.9 & 1.0 \\ 
  \hline
1.0 & 390 & 438 & 434 & 324 & 328 & 459 & 337 & 204 & 227 & 40 \\ 
  1.2 & 1,576 & 2,394 & 2,505 & 2,996 & 1,613 & 1,551 & 1,166 & 1,542 & 1,002 & 58 \\ 
  1.4 & 5,007 & 4,336 & 4,652 & 4,877 & 3,458 & 3,956 & 3,169 & 3,380 & 1,868 & 86 \\ 
  1.6 & 5,067 & 5,032 & 6,444 & 4,868 & 4,575 & 3,864 & 3,211 & 1,731 & 1,892 & 85 \\ 
  1.8 & 4,016 & 3,701 & 3,647 & 4,523 & 3,548 & 3,009 & 1,808 & 451 & 336 & 62 \\ 
  2.0 & 1,663 & 2,276 & 2,676 & 3,035 & 1,925 & 1,932 & 464 & 307 & 150 & 54 \\ 
  2.2 & 989 & 1,514 & 1,621 & 1,826 & 1,121 & 846 & 155 & 127 & 93 & 45 \\ 
  2.4 & 560 & 821 & 919 & 771 & 529 & 267 & 133 & 87 & 55 & 32 \\ 
  2.6 & 388 & 519 & 524 & 441 & 259 & 219 & 84 & 50 & 41 & 26 \\ 
  2.8 & 311 & 317 & 396 & 242 & 165 & 95 & 64 & 44 & 33 & 22 \\ 
  3.0 & 173 & 224 & 211 & 169 & 102 & 72 & 41 & 34 & 27 & 18 \\ 
  \end{tabular}

    \caption{Fribourg+R2C+C}
  \end{subtable}
  \hfill
  \begin{subtable}[t]{\subwidth\textwidth}
    \centering
    % latex table generated in R 3.1.2 by xtable 1.7-4 package
% Fri Jun 12 16:02:09 2015
\begin{tabular}{r|RRRRRRRRRR}
  & 0.1 & 0.2 & 0.3 & 0.4 & 0.5 & 0.6 & 0.7 & 0.8 & 0.9 & 1.0 \\ 
  \hline
1.0 & 126 & 118 & 97 & 60 & 51 & 52 & 62 & 36 & 48 & 30 \\ 
  1.2 & 432 & 517 & 345 & 262 & 160 & 126 & 92 & 120 & 109 & 40 \\ 
  1.4 & 1,044 & 331 & 133 & 89 & 45 & 22 & 19 & 31 & 27 & 20 \\ 
  1.6 & 358 & 24 & 11 & 5 & 4 & 6 & 5 & 3 & 3 & 4 \\ 
  1.8 & 19 & 5 & 1 & 1 & 1 & 1 & 1 & 1 & 1 & 1 \\ 
  2.0 & 1 & 1 & 1 & 1 & 1 & 1 & 1 & 1 & 1 & 1 \\ 
  2.2 & 1 & 1 & 1 & 1 & 1 & 1 & 1 & 1 & 1 & 1 \\ 
  2.4 & 1 & 1 & 1 & 1 & 1 & 1 & 1 & 1 & 1 & 1 \\ 
  2.6 & 1 & 1 & 1 & 1 & 1 & 1 & 1 & 1 & 1 & 1 \\ 
  2.8 & 1 & 1 & 1 & 1 & 1 & 1 & 1 & 1 & 1 & 1 \\ 
  3.0 & 1 & 1 & 1 & 1 & 1 & 1 & 1 & 1 & 1 & 1 \\ 
  \end{tabular}

    \caption{Fribourg+R}
  \end{subtable}

  \renewcommand{\tabcolsep}{0.05cm}
  \begin{subtable}[t]{\subwidth\textwidth}
    \centering
    % latex table generated in R 3.1.2 by xtable 1.7-4 package
% Fri Jun 12 16:02:09 2015
\begin{tabular}{r|RRRRRRRRRR}
  & 0.1 & 0.2 & 0.3 & 0.4 & 0.5 & 0.6 & 0.7 & 0.8 & 0.9 & 1.0 \\ 
  \hline
1.0 & 225 & 223 & 195 & 181 & 187 & 199 & 189 & 124 & 161 & 68 \\ 
  1.2 & 731 & 971 & 946 & 1,071 & 629 & 562 & 488 & 568 & 388 & 104 \\ 
  1.4 & 2,228 & 1,701 & 1,543 & 1,732 & 1,241 & 1,287 & 945 & 944 & 727 & 154 \\ 
  1.6 & 2,489 & 2,263 & 2,331 & 2,133 & 1,777 & 1,443 & 964 & 757 & 889 & 155 \\ 
  1.8 & 2,381 & 2,027 & 2,009 & 2,075 & 1,618 & 1,243 & 1,005 & 592 & 515 & 114 \\ 
  2.0 & 1,390 & 1,569 & 1,416 & 1,573 & 1,093 & 1,008 & 594 & 464 & 330 & 98 \\ 
  2.2 & 1,118 & 1,197 & 1,150 & 1,151 & 879 & 809 & 317 & 330 & 241 & 78 \\ 
  2.4 & 712 & 885 & 836 & 809 & 580 & 535 & 316 & 231 & 145 & 59 \\ 
  2.6 & 498 & 569 & 601 & 627 & 497 & 412 & 217 & 137 & 113 & 44 \\ 
  2.8 & 391 & 455 & 578 & 456 & 374 & 263 & 173 & 119 & 90 & 41 \\ 
  3.0 & 258 & 350 & 392 & 354 & 253 & 208 & 119 & 97 & 74 & 32 \\ 
  \end{tabular}

    \caption{Fribourg+M1}
  \end{subtable}
  \hfill
  \begin{subtable}[t]{\subwidth\textwidth}
    \centering
    % latex table generated in R 3.1.2 by xtable 1.7-4 package
% Fri Jun 12 16:02:09 2015
\begin{tabular}{r|RRRRRRRRRR}
  & 0.1 & 0.2 & 0.3 & 0.4 & 0.5 & 0.6 & 0.7 & 0.8 & 0.9 & 1.0 \\ 
  \hline
1.0 & 225 & 223 & 195 & 181 & 187 & 199 & 189 & 124 & 161 & 68 \\ 
  1.2 & 731 & 971 & 946 & 1,071 & 629 & 562 & 488 & 568 & 388 & 104 \\ 
  1.4 & 2,228 & 1,701 & 1,543 & 1,732 & 1,241 & 1,287 & 945 & 944 & 727 & 154 \\ 
  1.6 & 2,489 & 2,263 & 2,331 & 2,133 & 1,777 & 1,443 & 964 & 757 & 889 & 155 \\ 
  1.8 & 2,381 & 2,027 & 2,009 & 2,075 & 1,618 & 1,215 & 1,005 & 592 & 515 & 114 \\ 
  2.0 & 1,390 & 1,513 & 1,416 & 1,542 & 1,093 & 1,003 & 594 & 441 & 330 & 97 \\ 
  2.2 & 1,019 & 1,156 & 1,064 & 1,104 & 859 & 785 & 304 & 303 & 221 & 78 \\ 
  2.4 & 672 & 867 & 789 & 772 & 544 & 478 & 269 & 191 & 139 & 55 \\ 
  2.6 & 466 & 542 & 572 & 568 & 452 & 348 & 183 & 129 & 99 & 43 \\ 
  2.8 & 368 & 407 & 480 & 337 & 260 & 197 & 129 & 96 & 75 & 36 \\ 
  3.0 & 201 & 261 & 266 & 272 & 199 & 136 & 83 & 74 & 50 & 27 \\ 
  \end{tabular}

    \caption{Fribourg+M1+R2C}
  \end{subtable}

  \renewcommand{\tabcolsep}{0.05cm}
  \begin{subtable}[t]{\subwidth\textwidth}
    \centering
    % latex table generated in R 3.1.2 by xtable 1.7-4 package
% Wed Aug 19 09:24:32 2015
\begin{tabular}{r|RRRRRRRRRR}
  & 0.1 & 0.2 & 0.3 & 0.4 & 0.5 & 0.6 & 0.7 & 0.8 & 0.9 & 1.0 \\ 
  \hline
1.0 & 329 & 303 & 279 & 240 & 229 & 288 & 230 & 157 & 160 & 40 \\ 
  1.2 & 988 & 1,392 & 1,356 & 1,352 & 751 & 741 & 608 & 704 & 516 & 58 \\ 
  1.4 & 2,939 & 2,581 & 2,066 & 2,190 & 1,351 & 1,622 & 1,132 & 1,261 & 932 & 86 \\ 
  1.6 & 3,150 & 2,900 & 2,842 & 2,218 & 1,885 & 1,563 & 1,177 & 821 & 896 & 85 \\ 
  1.8 & 2,782 & 2,485 & 2,047 & 2,180 & 1,625 & 1,269 & 855 & 395 & 309 & 62 \\ 
  2.0 & 1,338 & 1,638 & 1,544 & 1,566 & 979 & 957 & 349 & 261 & 147 & 54 \\ 
  2.2 & 838 & 1,125 & 993 & 1,027 & 667 & 521 & 153 & 125 & 93 & 45 \\ 
  2.4 & 494 & 700 & 624 & 524 & 296 & 214 & 126 & 87 & 55 & 32 \\ 
  2.6 & 327 & 434 & 383 & 334 & 212 & 163 & 82 & 50 & 41 & 26 \\ 
  2.8 & 283 & 273 & 305 & 202 & 144 & 95 & 60 & 44 & 33 & 22 \\ 
  3.0 & 164 & 200 & 173 & 142 & 92 & 72 & 41 & 34 & 27 & 18 \\ 
  \end{tabular}

    \caption{Fribourg+M1+R2C+C}
  \end{subtable}
  \hfill
  \begin{subtable}[t]{\subwidth\textwidth}
    \centering
    % latex table generated in R 3.1.2 by xtable 1.7-4 package
% Wed Aug 19 09:24:32 2015
\begin{tabular}{r|RRRRRRRRRR}
  & 0.1 & 0.2 & 0.3 & 0.4 & 0.5 & 0.6 & 0.7 & 0.8 & 0.9 & 1.0 \\ 
  \hline
1.0 & 215 & 213 & 189 & 174 & 175 & 192 & 186 & 121 & 156 & 68 \\ 
  1.2 & 712 & 914 & 913 & 1,075 & 619 & 563 & 526 & 620 & 416 & 104 \\ 
  1.4 & 2,075 & 1,620 & 1,503 & 1,650 & 1,254 & 1,339 & 1,003 & 1,006 & 848 & 154 \\ 
  1.6 & 2,344 & 2,062 & 2,340 & 2,016 & 1,755 & 1,520 & 1,053 & 858 & 986 & 155 \\ 
  1.8 & 2,205 & 1,873 & 1,920 & 2,040 & 1,689 & 1,315 & 1,080 & 664 & 598 & 114 \\ 
  2.0 & 1,290 & 1,485 & 1,405 & 1,522 & 1,134 & 1,044 & 652 & 531 & 392 & 98 \\ 
  2.2 & 1,023 & 1,119 & 1,092 & 1,127 & 868 & 875 & 376 & 359 & 262 & 78 \\ 
  2.4 & 674 & 849 & 790 & 807 & 617 & 544 & 355 & 251 & 156 & 59 \\ 
  2.6 & 478 & 549 & 594 & 597 & 510 & 431 & 231 & 147 & 116 & 44 \\ 
  2.8 & 370 & 439 & 559 & 455 & 382 & 283 & 182 & 124 & 93 & 41 \\ 
  3.0 & 249 & 341 & 388 & 348 & 260 & 225 & 123 & 101 & 77 & 32 \\ 
  \end{tabular}

    \caption{Fribourg+M1+M2}
  \end{subtable}
\end{scriptsize}
\caption{Median complement sizes based on the 10,939 effective samples of the internal tests for each of the 110 transition density/acceptance density classes of the \goal{} test set.}
\label{i.g.matrices}
\end{figure}

% Reset table spacings to our default values
\tablestyle
