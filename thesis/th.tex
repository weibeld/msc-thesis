%\documentclass[a4paper]{report}
%\usepackage{a4wide}
\documentclass{scrreprt}

% Allow direct typing of ä, ö, ü, for example in "Büchi"
\usepackage[utf8]{inputenc}

\newcommand{\om}{{$\omega$}}

\begin{document}
% Title page
\title{Complementation of Non-Deterministic B\"uchi Automata}
\author{Daniel Weibel}
\date{\today}
\maketitle

% Abstract
% \begin{abstract}
% Hello world
% \end{abstract}

% Table of contents
\tableofcontents

\chapter{Introduction}

% Context of study
A Büchi complementation construction takes as input a Büchi automaton $A$ and produces as output another Büchi automaton $B$ which accepts the complement language of the input automaton $A$. Complement language denotes the ``contrary'' language, that is, $B$ must \emph{accept} (over a given alphabet) every word that $A$ \emph{does not} accept, and must in turn \emph{not accept} every word that $A$ \emph{accepts}.

Büchi automata are finite automata (that is, having a finite number of states) which operate on infinite words (that is, words that ``never end''). Operating on infinite words, they belong thus to the category \om-automata. An important application of Büchi automata is in model checking which is a formal system verification technique. There, they are used to represent both, the description of the system to be checked for the presence of a correctness property, and (the negation of) this correctness property itself.

In one approach to model checking, the correctness property is directly specified as a Büchi automaton
One approach to model checking requires that the Büchi automaton representing the correctness property is complemented. It is here that the problem of Büchi complementation has one of its practical applications. 

% Stating the problem, reason the research is worth tackling
The complementation of non-deterministic Büchi automata is hard. It has been proven to be intrinsically exponential in the number of generated states [cite]. That is, the number of states of the output automaton is, in the worst case, an exponential function of the number of states of the input automaton. However, since the introduction of Büchi automata in the 1960's, significant process in reducing the complexity (in other words, the degree of exponentiality) of the Büchi complementation problem has been made. Some numbers [list complexities of the different constructions]. 

% Aim and scope

% Significance: advance knowledge, contribute to solutin of practical problem, novel use of a procedure or technique?

% Overview

\chapter{The Wider Context: System Verification}

\chapter{The B\"uchi Complementation Problem}

\chapter{The Fribourg Construction}

\chapter{Discussion}

\chapter{Conclusions}

\end{document}